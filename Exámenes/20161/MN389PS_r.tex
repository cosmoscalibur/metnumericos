\documentclass[12pt]{article}
\usepackage[letterpaper,margin={1.5cm}]{geometry}
\usepackage{amsmath, amssymb, amsfonts}
\usepackage[utf8]{inputenc}
\usepackage[T1]{fontenc}
\usepackage[spanish]{babel}
\usepackage{tikz}
\usepackage{graphicx,enumitem}
\usepackage{multicol}
\setlength{\marginparsep}{12pt} \setlength{\marginparwidth}{0pt} \setlength{\headsep}{.8cm} \setlength{\headheight}{15pt} \setlength{\labelwidth}{0mm} \setlength{\parindent}{0mm} \renewcommand{\baselinestretch}{1.15} \setlength{\fboxsep}{5pt} \setlength{\parskip}{3mm} \setlength{\arraycolsep}{2pt}
\renewcommand{\sin}{\operatorname{sen}}
\newcommand{\N}{\ensuremath{\mathbb{N}}}
\newcommand{\Z}{\ensuremath{\mathbb{Z}}}
\newcommand{\Q}{\ensuremath{\mathbb{Q}}}
\newcommand{\R}{\ensuremath{\mathbb{R}}}
\newcommand{\C}{\ensuremath{\mathbb{C}}}
\newcommand{\I}{\ensuremath{\mathbb{I}}}
\graphicspath{{../imagenes/}{imagenes/}{..}}
\allowdisplaybreaks{}
\raggedbottom{}
\setlength{\topskip}{0pt plus 2pt}
\newcommand{\profesor}{Edward Y. Villegas}
\newcommand{\asignatura}{M\'ETODOS NUM\'ERICOS}
\newcommand{\diff}[3]{\frac{d^{#3} #1}{d#2^{#3}}}
\newcommand{\pdiff}[3]{\frac{\partial^{#3} #1}{\partial#2^{#3}}}
\newcommand{\abs}[1]{\left| #1 \right|}
\begin{document}
  \pagestyle{empty}
  \begin{minipage}{\linewidth}
    \centering
    \begin{tikzpicture}[very thick,font=\small]
      \node at (2,6) {\includegraphics[width=3.5cm]{logoudem}};
      \node at (9.5,6) {\includegraphics[width=9cm]{cbudem}};
      \node[fill=white,draw=white,inner sep=1mm] at (9.5,5.05) {\bf Permanencia con calidad, Acompa\~nar para exigir};
      \node[fill=white,draw=white,inner sep=1mm] at (7.5,4.2) {\Large\bf DEPARTAMENTO DE CIENCIAS B\'ASICAS};
      \draw (0,0) rectangle (18,3.5);
      \draw (0,2.5)--(18,2.5) (0,1.5)--(18,1.5) (15,4.2)--(18,4.2) node[below,pos=.5] {CALIFICACI\'ON} (15,2.5)--(15,7)--(18,7)--(18,3.5) (8.4,0)--(8.4,1.5) (15,0)--(15,1.5) (10,1.5)--(10,2.5);
      \node[right] at (0,3.2) {\bf Alumno:}; \node[right] at (15,3.2) {\bf Carn\'e:};
      \node[right] at (0,2.2) {\bf Asignatura:};
      \node at (6,1.95) {\asignatura};
      \node[right] at (10,2.2) {\bf Profesor:};
      \node at (15,1.95) {\profesor};
      \node[right] at (0,1.2) {\bf Examen:};
      \draw (3.8,.9) rectangle (4.4,1.3); \node[left] at (3.8,1.1) {Parcial:};
      \draw (3.8,.2) rectangle (4.4,.6); \node[left] at (3.8,.4) {Previa:};
      \draw (7.4,.9) rectangle (8,1.3); \node[left] at (7.4,1.1) {Final:};
      \draw (7.4,.2) rectangle (8,.6); \node[left] at (7.4,.4) {Habilitaci\'on:};
       \node at (4,1.1) {X}; % Parcial
      \node[right] at (10,.5) {3 - 8 - 9}; % Número de grupo
      \node[right] at (10,1.) {30 Marzo de 2016}; % Fecha de presentación
      \node[right] at (8.4,1.05) {\bf Fecha:}; \node[right] at (8.4,.45) {\bf Grupo:};
      \node[align=center,text width=3cm,font=\footnotesize] at (16.5,.75) {\centering\bf Use solo tinta\\y escriba claro};
    \end{tikzpicture}
  \end{minipage}
Para el desarrollo de los cálculos puede usar \textbf{exclusivamente calculadora} (no se permite el uso de portátil o celulares en el examen), y % {presencial}
al final del examen, encontrará una \textbf{hoja de formulas} para su ayuda. %{Parcial - Final}
Todo valor reportado debe aproximarse a \textbf{5 CIFRAS SIGNIFICATIVAS} con \textbf{REDONDEO SIMÉTRICO} (no es necesario en enteros)% {Todos}
. Recuerde el uso del separador decimal acorde a la \textbf{NTC} % {Todos}
.
Indique \textbf{clara y explícitamente la respuesta final} de cada pregunta % {Todos}
en \textbf{lapicero} (requisito en caso de reclamación) % {presencial}
, y \textbf{justifique todas sus respuestas y procedimientos. Si algún elemento solicitado, en teoría no puede realizarse, indíquelo y justifique por que no se puede realizar lo solicitado. % {Todos}
Si necesita espacio adicional, use el respaldo de las hojas para el procedimiento (no se permiten hojas extras)}. % {presencial}
\vspace{-.5cm}
  \begin{enumerate}[leftmargin=*,widest=9]
     \item Se ha realizado un registro de temperatura cada 6 horas, con la toma de datos siguiente para aproximar la temperatura en horas no registradas.
     \begin{equation*}
     \begin{array}{|c|c|}
     \hline
     \text{Hora del día} & \text{Temperatura (} ^{\circ}C \text{)}\\
     \hline
     6 & 17\\
     12 & 21\\
     18 & 22\\
     \hline
     \end{array}
     \end{equation*}
     Use la información completa de la tabla para responder.
     \begin{enumerate}[label=\alph*]
     \item (\(0.4\)) ¿Que método de interpolación deberá usar para aproximar de la temperatura en horas que no se registraron? Justifique.
     \vspace{4cm}
     \item (\(1.2\)) Obtenga el polinomio interpolante correspondiente al método que indico en el literal anterior usando todos los datos. Detalle el procedimiento.
     \vspace*{10cm}
     \item (\(0.4\)) Aproxime el valor de temperatura a las 8:30 A.M.
     \vspace{2cm}
     \end{enumerate}
    \item Algoritmo de Horner.
    \begin{enumerate}[label=\alph*]
    \item (\(0.4\)) Dada la factorización de Horner \( P(x) = Q(x)(x-x_0) + b_0\), demuestre que \(P^{\prime}(x_0) = Q(x_0)\).
    \vspace{4cm}
    \item (\(1.0\)) Dado \(P_2(x) = -6 + 4x^2 - 5x\), use el algoritmo de Horner para evaluar \(x_0 = 2\). ¿Es raíz?
    \vspace{4cm}
    \item (\(0.4\)) Si el literal anterior fue raíz, indique con ayuda del procedimiento anterior cual es el polinomio de grado 1 que posee la siguiente raíz (polinomio cociente). Si no es raíz, realice 3 iteraciones de Newton-Horner para encontrar la aproximación a la raíz con \(x_0 = 2\) como valor inicial.
    \vspace{2cm}
   \end{enumerate}
   \item (\(1.2\)) Dado \(P_2(x) = -6 + 4x^2 - 5x\), con \(x_0 = 2.6\), use un método abierto para aproximar una de las raíces del polinomio con una tolerancia en \(x\) de \(0.01\).
  \vspace{10cm}
  \end{enumerate}
\begin{center}
\textbf{Hoja de fórmulas}
\end{center}
{\large
\[
\begin{array}{cc}
x_{i+1} = g(x_i) \qquad & \qquad |g^\prime(x_i)| < 1 \\
x_{i+1} = x_i - \frac{f(x_i)}{f^\prime(x_i)} \qquad & \qquad x_{i+1} = x_i - \frac{P(x_i)}{Q(x_i)} \\
x_{i+1} = x_i - \frac{f(x_i) \Delta x}{f(x_i + \Delta x) - f(x_i)} \qquad & \qquad x_{i+2} = x_{i+1} - \frac{f(x_{i+1}) (x_{i+1}-x_i)}{f(x_{i+1}) - f(x_i)} \\
b_n = a_n \qquad & \qquad
b_k = b_{k+1}x_0 + a_k \\
L_{n, i}(x) = \prod\limits_{\substack{j=0\\ i \neq j}}^n \frac{x - x_j}{x_i - x_j} \qquad & \qquad
P_n(x) = \sum\limits_{i = 0}^n f(x_i)L_{n,i}(x) \\
f\left[x_i, x_{i+1}\right] = \frac{f(x_{i+1})-f(x_i)}{x_{i+1}-x_i} \qquad & \qquad
f\left[ x_i, x_{i+1}, \ldots, x_{j-1}, x_j\right] = \frac{f\left[x_{i+1}, \ldots, x_{j-1}, x_j\right] - f\left[ x_i, x_{i+1}, \ldots, x_{j-1} \right]}{x_j - x_i} \\
z_{2i} = z_{2i+1} = x_i \qquad & \qquad
H_{2n+1} = f(x_0) + \sum\limits_{k=1}^{2n+1} f\left[x_0, \ldots, x_k\right] \prod\limits_{i = 0}^{k-1}(x-x_i) % \\
\end{array}
\]
}
\end{document}
