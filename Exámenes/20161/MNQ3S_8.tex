\documentclass[12pt]{article}
\usepackage[letterpaper,margin={1.5cm}]{geometry}
\usepackage{amsmath, amssymb, amsfonts}
\usepackage[utf8]{inputenc}
\usepackage[T1]{fontenc}
\usepackage[spanish]{babel}
\usepackage{tikz}
\usepackage{graphicx,enumitem}
\usepackage{multicol}
\setlength{\marginparsep}{12pt} \setlength{\marginparwidth}{0pt} \setlength{\headsep}{.8cm} \setlength{\headheight}{15pt} \setlength{\labelwidth}{0mm} \setlength{\parindent}{0mm} \renewcommand{\baselinestretch}{1.15} \setlength{\fboxsep}{5pt} \setlength{\parskip}{3mm} \setlength{\arraycolsep}{2pt}
\renewcommand{\sin}{\operatorname{sen}}
\newcommand{\N}{\ensuremath{\mathbb{N}}}
\newcommand{\Z}{\ensuremath{\mathbb{Z}}}
\newcommand{\Q}{\ensuremath{\mathbb{Q}}}
\newcommand{\R}{\ensuremath{\mathbb{R}}}
\newcommand{\C}{\ensuremath{\mathbb{C}}}
\newcommand{\I}{\ensuremath{\mathbb{I}}}
\graphicspath{{../imagenes/}{imagenes/}{..}}
\allowdisplaybreaks{}
\raggedbottom{}
\setlength{\topskip}{0pt plus 2pt}
\newcommand{\profesor}{Edward Y. Villegas}
\newcommand{\asignatura}{M\'ETODOS NUM\'ERICOS}
\newcommand{\diff}[3]{\frac{d^{#3} #1}{d#2^{#3}}}
\newcommand{\pdiff}[3]{\frac{\partial^{#3} #1}{\partial#2^{#3}}}
\newcommand{\abs}[1]{\left| #1 \right|}
\begin{document}
  \pagestyle{empty}
  \begin{minipage}{\linewidth}
    \centering
    \begin{tikzpicture}[very thick,font=\small]
      \node at (2,6) {\includegraphics[width=3.5cm]{logoudem}};
      \node at (9.5,6) {\includegraphics[width=9cm]{cbudem}};
      \node[fill=white,draw=white,inner sep=1mm] at (9.5,5.05) {\bf Permanencia con calidad, Acompa\~nar para exigir};
      \node[fill=white,draw=white,inner sep=1mm] at (7.5,4.2) {\Large\bf DEPARTAMENTO DE CIENCIAS B\'ASICAS};
      \draw (0,0) rectangle (18,3.5);
      \draw (0,2.5)--(18,2.5) (0,1.5)--(18,1.5) (15,4.2)--(18,4.2) node[below,pos=.5] {CALIFICACI\'ON} (15,2.5)--(15,7)--(18,7)--(18,3.5) (8.4,0)--(8.4,1.5) (15,0)--(15,1.5) (10,1.5)--(10,2.5);
      \node[right] at (0,3.2) {\bf Alumno:}; \node[right] at (15,3.2) {\bf Carn\'e:};
      \node[right] at (0,2.2) {\bf Asignatura:};
      \node at (6,1.95) {\asignatura};
      \node[right] at (10,2.2) {\bf Profesor:};
      \node at (15,1.95) {\profesor};
      \node[right] at (0,1.2) {\bf Examen:};
      \draw (3.8,.9) rectangle (4.4,1.3); \node[left] at (3.8,1.1) {Parcial:};
      \draw (3.8,.2) rectangle (4.4,.6); \node[left] at (3.8,.4) {Previa:};
      \draw (7.4,.9) rectangle (8,1.3); \node[left] at (7.4,1.1) {Final:};
      \draw (7.4,.2) rectangle (8,.6); \node[left] at (7.4,.4) {Habilitaci\'on:};
       \node at (4, .4) {X}; % Quiz
      \node[right] at (10,.5) {8}; % Número de grupo
      \node[right] at (10,1.) {15 de abril de 2016}; % Fecha de presentación
      \node[right] at (8.4,1.05) {\bf Fecha:}; \node[right] at (8.4,.45) {\bf Grupo:};
      \node[align=center,text width=3cm,font=\footnotesize] at (16.5,.75) {\centering\bf Use solo tinta\\y escriba claro};
    \end{tikzpicture}
  \end{minipage}
Para el desarrollo de los cálculos puede usar \textbf{exclusivamente calculadora} (no se permite el uso de portátil o celulares en el examen), y % {presencial}
puede disponer de sus apuntes de clase. % {Quiz}
Todo valor reportado debe aproximarse a \textbf{5 CIFRAS SIGNIFICATIVAS} con \textbf{REDONDEO SIMÉTRICO} (no es necesario en enteros y valores dados en enunciado)% {Todos}
. Recuerde el uso del separador decimal acorde a la \textbf{NTC} % {Todos}
.
Indique \textbf{clara y explícitamente la respuesta final} de cada pregunta % {Todos}
en \textbf{lapicero} (requisito en caso de reclamación) % {presencial}
, y \textbf{justifique todas sus respuestas y procedimientos. Si algún elemento solicitado, en teoría no puede realizarse, indíquelo y justifique por que no se puede realizar lo solicitado. % {Todos}
}.
\vspace{-.5cm}
  \begin{enumerate}[leftmargin=*,widest=9]
     \item \(2.0\) Dados los puntos siguientes en representación de una función continua, aproxime \(f^\prime(2)\) con la forma \(O(h^2)\) adecuada.
     \[
     \begin{array}{lr}
     x  & f(x)\\
     1.8 \quad & 0.94164\\
     1.9 & 0.98510\\
     2.1 & 0.98510\\
     2.2 & 0.94154
     \end{array}
     \]

\textbf{R/} Podemos usar puntos adelante y atrás del solicitado con la separación indicada, y así aplicar un esquema central de 3 puntos para la primera derivada.
    \[
    f^\prime(2) = \frac{f(2.2000) - f(1.8000)}{2(0.2)} = \frac{0.94164 - 0.94164}{0.40000} = 0
    \]
    \hspace{5cm} o
    \[
    f^\prime(2) = \frac{f(2.1000) - f(1.9000)}{2(0.1)} = \frac{0.98510 - 0.98510}{0.20000} = 0
    \]
    \item \(2.0\) Dados los puntos que representan una función continua
    \[
    \begin{array}{lr}
    x & f(x)\\
    4 & 8\\
    6 & 9\\
    8 & 9\\
    10 & 10   
    \end{array}
    \]
    Determine el valor de la integral \( \int^{10}_4 f(x)dx\), usando criterio de mayor precisión.

\textbf{R/} Primero, nótese que la separación entre los puntos es la misma, por lo cual aplican los métodos para puntos equidistantes (\(h=2\)). La mayor precisión siempre corresponde al uso de la aproximación de mayor grado. En este caso, el número de intervalos es justamente 3, que es el mínimo requerido por Simpson \(3/8\) (3 intervalos).
\begin{eqnarray*}
\int^{12}_4 f(x)dx  &=& \frac{3}{8}(2)\left(f(4) + f(10) + 3 (f(6) + f(8)) \right)\\
&=&\frac{3}{8}(2)\left(8 + 10 + 3 (9 + 9) \right) \\
&=&54
\end{eqnarray*}
   \item (\(1.0\)) ¿Es posible usar solo hexágonos para mallar un dominio bidimensional para integración sobre elementos finitos? Justifique.

\textbf{R/} Si es posible usar solo hexágonos. Para los mallados en problemas de integración sobre elementos finitos se usan triángulos (generalmente), cuadriláteros y hexágonos (este último poco común) por su propiedad de ser formas teselables, que corresponde que simétricamente llenan con facilidad un dominio arbitrario y a que llevan a formas con propiedades idóneas en las funciones interpolantes.
  \end{enumerate}
\end{document}
