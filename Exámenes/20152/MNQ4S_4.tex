\documentclass[12pt]{article}
\usepackage[letterpaper,margin={1.5cm}]{geometry}
\usepackage{amsmath, amssymb, amsfonts}
\usepackage[utf8]{inputenc}
\usepackage[T1]{fontenc}
\usepackage[spanish]{babel}
\usepackage{tikz}
\usepackage{graphicx,enumitem}
\usepackage{multicol}
\setlength{\marginparsep}{12pt} \setlength{\marginparwidth}{0pt} \setlength{\headsep}{.8cm} \setlength{\headheight}{15pt} \setlength{\labelwidth}{0mm} \setlength{\parindent}{0mm} \renewcommand{\baselinestretch}{1.15} \setlength{\fboxsep}{5pt} \setlength{\parskip}{3mm} \setlength{\arraycolsep}{2pt}
\renewcommand{\sin}{\operatorname{sen}}
\newcommand{\N}{\ensuremath{\mathbb{N}}}
\newcommand{\Z}{\ensuremath{\mathbb{Z}}}
\newcommand{\Q}{\ensuremath{\mathbb{Q}}}
\newcommand{\R}{\ensuremath{\mathbb{R}}}
\newcommand{\C}{\ensuremath{\mathbb{C}}}
\newcommand{\I}{\ensuremath{\mathbb{I}}}
\graphicspath{{../imagenes/}{imagenes/}{..}}
\allowdisplaybreaks{}
\raggedbottom{}
\setlength{\topskip}{0pt plus 2pt}
\newcommand{\profesor}{Edward Y. Villegas}
\newcommand{\asignatura}{M\'ETODOS NUM\'ERICOS}
\newcommand{\diff}[3]{\frac{d^{#3} #1}{d#2^{#3}}}
\newcommand{\pdiff}[3]{\frac{\partial^{#3} #1}{\partial#2^{#3}}}
\newcommand{\abs}[1]{\left| #1 \right|}
\begin{document}
  \pagestyle{empty}
  \begin{minipage}{\linewidth}
    \centering
    \begin{tikzpicture}[very thick,font=\small]
      \node at (2,6) {\includegraphics[width=3.5cm]{logoudem}};
      \node at (9.5,6) {\includegraphics[width=9cm]{cbudem}};
      \node[fill=white,draw=white,inner sep=1mm] at (9.5,5.05) {\bf Permanencia con calidad, Acompa\~nar para exigir};
      \node[fill=white,draw=white,inner sep=1mm] at (7.5,4.2) {\Large\bf DEPARTAMENTO DE CIENCIAS B\'ASICAS};
      \draw (0,0) rectangle (18,3.5);
      \draw (0,2.5)--(18,2.5) (0,1.5)--(18,1.5) (15,4.2)--(18,4.2) node[below,pos=.5] {CALIFICACI\'ON} (15,2.5)--(15,7)--(18,7)--(18,3.5) (8.4,0)--(8.4,1.5) (15,0)--(15,1.5) (10,1.5)--(10,2.5);
      \node[right] at (0,3.2) {\bf Alumno:}; \node[right] at (15,3.2) {\bf Carn\'e:};
      \node[right] at (0,2.2) {\bf Asignatura:};
      \node at (6,1.95) {\asignatura};
      \node[right] at (10,2.2) {\bf Profesor:};
      \node at (15,1.95) {\profesor};
      \node[right] at (0,1.2) {\bf Examen:};
      \draw (3.8,.9) rectangle (4.4,1.3); \node[left] at (3.8,1.1) {Parcial:};
      \draw (3.8,.2) rectangle (4.4,.6); \node[left] at (3.8,.4) {Previa:};
      \draw (7.4,.9) rectangle (8,1.3); \node[left] at (7.4,1.1) {Final:};
      \draw (7.4,.2) rectangle (8,.6); \node[left] at (7.4,.4) {Habilitaci\'on:};
      \node at (4, .4) {X}; % Quiz
      \node[right] at (10,.5) {4}; % Número de grupo
      \node[right] at (10,1.) {13 de noviembre de 2015}; % Fecha de presentación
      \node[right] at (8.4,1.05) {\bf Fecha:}; \node[right] at (8.4,.45) {\bf Grupo:};
      \node[align=center,text width=3cm,font=\footnotesize] at (16.5,.75) {\centering\bf Use solo tinta\\y escriba claro};
    \end{tikzpicture}
  \end{minipage}
Indique \textbf{clara y explícitamente la respuesta final} de cada pregunta en \textbf{lapicero}, y \textbf{justifique} todas sus respuestas y procedimientos. Si algún elemento solicitado, en teoría no puede realizarse, indíquelo y justifique por que no se puede realizar lo solicitado. Respuestas sin justificación no cuentan.
  \begin{enumerate}[leftmargin=*,widest=9]
    \item Sea
    \[
    A = \begin{pmatrix}
    -4 & 4 & 0\\ 2 & 8 & -5\\ 0 & 2 & 5
    \end{pmatrix}
    \]     
    \begin{enumerate}[label=\alph*]
    \item (\(1.0\)) ¿ Es \(A\) estrictamente diagonal dominante?
    Se observa en la primera fila de \(A\), que el valor absoluto del elemento de la diagonal no es mayor estricto que la suma del valor absoluto de los elementos de la misma fila ( \(|-4| = |4|+|0|\)). Por ende, no es estrictamente diagonal dominante.
    \item (\(1.0\)) ¿Es \(A\) definida positiva?
Una de las propiedades de las matrices definidas positivas, es que los elementos de la diagonal son mayores que cero. Se observa que el primer elemento diagonal es negativo (\(-4\)), por ende no es una matriz definida positiva.
\textit{Nota: Recordar que las propiedades o ejemplos numéricos solo sirven para mostrar falsedad, como en este caso.}
    \end{enumerate}
    \item Sea
    \[
    B = \begin{pmatrix}
    0.2 & 1 \\ 0.1 & 0.8
    \end{pmatrix}, \qquad \vec{v} = \begin{pmatrix}
    1 \\ 1
    \end{pmatrix}
    \] 
    \begin{enumerate}[label=\alph*]
    \item (\(1.0\)) Calcule \(\rho (B)\).
    \begin{eqnarray*}
    p(\lambda) &=& \det(B - \lambda\I)\\
    & = & \begin{vmatrix}
    0.2-\lambda & 1\\ 0.1 & 0.8-\lambda
    \end{vmatrix} \\
    & = & (0.2-\lambda)(0.8-\lambda) - 1(0.1)\\
    & = & 0.16 -0.8\lambda -0.2\lambda + \lambda^2 -0.1\\
    & = & \lambda^2 - \lambda + 0.06 = 0\\
    \lambda & = & 0.50 \pm 0.44\\
    \lambda_1 = 0.06, \qquad \lambda_2 = 0.94\\
    \rho(B) & = & 0.94
    \end{eqnarray*}
    \item (\(1.0\)) ¿Es \( \lim\limits_{k \rightarrow \infty}B^k \vec{v}\) igual o diferente de \(\vec{0}\) (vector nulo)?
    Dado que \(\rho(B) < 1\), sabemos que \(B\) es una matriz convergente. Luego, sus sucesivas potencias tienden a la matriz nula, y al multiplicar por el vector, será el vector nulo.
    \begin{eqnarray*}
    \lim\limits_{k \rightarrow \infty}B^k \vec{v} & = & \lim\limits_{k \rightarrow \infty}\begin{pmatrix}
    0.2 & 1 \\ 0.1 & 0.8
    \end{pmatrix}^k\begin{pmatrix}
    1 \\ 1
    \end{pmatrix} \\
    & = & \begin{pmatrix}
    0.0 & 0.0 \\ 0.0 & 0.0
    \end{pmatrix}\begin{pmatrix}
    1 \\ 1
    \end{pmatrix} \\
    & = & \begin{pmatrix}
    0 \\ 0
    \end{pmatrix}
    \end{eqnarray*}
    \end{enumerate}
  \item (\(1.0\)) Si \(T\) es una matriz convergente, demuestre que \(T^2\) es convergente. (Recordar que \(T^2 = P^T S^2 P\)).
  De la definición de matriz convergente y de radio espectral, tenemos que 
  \[ 0 \leq |\lambda_i| \leq \rho(T) < 1.\]
  Para elevar a una potencia una matriz, usamos las matrices similares, que son matrices diagonales formadas por los autovalores de la matriz de interes. De esta forma
  \begin{eqnarray*}
  T & = & P^T \begin{pmatrix}
  \lambda_1 & 0\\ 0 & \lambda_2
  \end{pmatrix} P\\
  T^2 & = & P^T \begin{pmatrix}
  \lambda_1 & 0\\ 0 & \lambda_2
  \end{pmatrix}^2 P\\
  & = & P^T \begin{pmatrix}
  \lambda_1^2 & 0\\ 0 & \lambda_2^2
  \end{pmatrix} P
  \end{eqnarray*}
  De la última ecuación vemos que los autovalores de \(T^2\) son los cuadrados de los autovalores de \(T\), de manera que si construimos el cuadrado de la relación inicial
  \begin{eqnarray*}
  0 \leq |\lambda_i| &\leq& \rho(T) < 1 \\
  0^2 \leq |\lambda_i|^2 &\leq& \rho(T)^2 < 1^2\\
  0 \leq |\lambda_i|^2 = |\lambda_i^2| &\leq& \rho(T^2) = \rho(T)^2 < 1
  \end{eqnarray*}
  De esta última relación se observa que si el radio espectral de \(T\) es menor que \(1\), el radio espectral de \(T^2\) lo será también.
  * (\(0.8\)) 
  Si \(A\) es definida positiva, implica que 
  \begin{eqnarray*}
  A &=& A^T\\
  \vec{x}^T A \vec{x} &>& 0
  \end{eqnarray*}
  Ahora, con \(-A\)
  \begin{eqnarray*}
  (-A)^T = -(A^T) = -A  \\
  \vec{x}^T (-A) \vec{x} = -(\vec{x}^T A \vec{x}) < 0
  \end{eqnarray*}
  Luego, \(-A\) no es definida positiva.
  \end{enumerate}
\clearpage
\begin{center}
\textbf{Hoja de fórmulas}
\end{center}
{\large
\[
\begin{array}{cc}
h = \frac{b - a}{n} \qquad & \qquad
x_{i+2} = \frac{f(x_{i+1}){x_i}-f(x_{i}){x_{i+1}}}{f(x_{i+1}) - f(x_i)} \\
L_{n, i}(x) = \prod\limits_{\substack{j=0\\ i \neq j}}^n \frac{x - x_j}{x_i - x_j} \qquad & \qquad
P_n(x) = \sum\limits_{i = 0}^n f(x_i)L_{n,i}(x) \\
f\left[x_i, x_{i+1}\right] = \frac{f(x_{i+1})-f(x_i)}{x_{i+1}-x_i} \qquad & \qquad
f\left[ x_i, x_{i+1}, \ldots, x_{j-1}, x_j\right] = \frac{f\left[x_{i+1}, \ldots, x_{j-1}, x_j\right] - f\left[ x_i, x_{i+1}, \ldots, x_{j-1} \right]}{x_j - x_i} \\
z_{2i} = z_{2i+1} = x_i \qquad & \qquad
H_{2n+1} = f(x_0) + \sum\limits_{k=1}^{2n+1} f\left[x_0, \ldots, x_k\right] \prod\limits_{i = 0}^{k-1}(x-x_i)  \\
f^\prime(x_i) \approx \frac{f(x_{i+1}) - f(x_{i-1})}{2h} \qquad & \qquad
I = \frac{h}{3}\left( f(x_0) + f(x_n) + 2\sum\limits_{i=1}^{n/2-1}f(x_{2i}) + 4\sum\limits_{i=0}^{n/2-1}f(x_{2i+1}) \right) \\
f^\prime(x_i) = \frac{f(x_{i+1})-f(x_i)}{h} \qquad & \qquad
I = \frac{h}{2}\left( f(x_0) + f(x_n) + 2\sum\limits_{i = 1}^{n-1}f(x_i)\right) \\
\abs{\pdiff{f(t,y)}{y}{}} \leq L & \abs{f(t, y_1) -f(t, y_0)} \leq L\abs{y_1 - y_0}\\
W_{i+1} = y_i + W f(t_i,W_i) & W_{i+1} = W_i + h f(t_i,W_i) + \frac{h^2}{2} \left.\diff{f(t,W)}{t}{} \right|_{t_i, W_i} \\
k_1  =  h f(t_i,W_i) & k_2  =  h f(t_i+h/2,W_i + k_1/2) \\
k_4  =  h f(t_{i+1},W_i + k_3) & k_3  =  h f(t_i+h/2,W_i + k_2/2)\\
\vec{x}^TA\vec{x} > 0 & W_{i+1} = W_i + \frac{k_1+2k_2+2k_3+k_4}{6}\\
A = A^T & A = D - L - U\\
|a_{ii}| > \sum\limits_{\substack{j=0\\j\neq i}}^n |a_{ij}| & \vec{x}^{(k+1)} = T\vec{x}^{(k)} + \vec{c}\\
T_J = D^{-1}(L+U) & p(\lambda) = \det(A-\lambda \I) = 0\\
\vec{c}_J = D^{-1}\vec{b} & \rho(A) = \max\limits_{1\leq i\leq n}|\lambda_i|\\
T_G = (D-L)^{-1}U & \vec{c}_G = (D-L)^{-1}\vec{b}
\end{array}
\]
}
\end{document}
