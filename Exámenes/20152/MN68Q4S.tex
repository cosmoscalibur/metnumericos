\documentclass[12pt]{article}
\usepackage[letterpaper,margin={1.5cm}]{geometry}
\usepackage{amsmath, amssymb, amsfonts}
\usepackage[utf8]{inputenc}
\usepackage[T1]{fontenc}
\usepackage[spanish]{babel}
\usepackage{tikz}
\usepackage{graphicx,enumitem}
\usepackage{multicol}
\setlength{\marginparsep}{12pt} \setlength{\marginparwidth}{0pt} \setlength{\headsep}{.8cm} \setlength{\headheight}{15pt} \setlength{\labelwidth}{0mm} \setlength{\parindent}{0mm} \renewcommand{\baselinestretch}{1.15} \setlength{\fboxsep}{5pt} \setlength{\parskip}{3mm} \setlength{\arraycolsep}{2pt}
\renewcommand{\sin}{\operatorname{sen}}
\newcommand{\N}{\ensuremath{\mathbb{N}}}
\newcommand{\Z}{\ensuremath{\mathbb{Z}}}
\newcommand{\Q}{\ensuremath{\mathbb{Q}}}
\newcommand{\R}{\ensuremath{\mathbb{R}}}
\newcommand{\C}{\ensuremath{\mathbb{C}}}
\newcommand{\I}{\ensuremath{\mathbb{I}}}
\graphicspath{{../imagenes/}{imagenes/}{..}}
\allowdisplaybreaks{}
\raggedbottom{}
\setlength{\topskip}{0pt plus 2pt}
\newcommand{\profesor}{Edward Y. Villegas}
\newcommand{\asignatura}{M\'ETODOS NUM\'ERICOS}
\newcommand{\diff}[3]{\frac{d^{#3} #1}{d#2^{#3}}}
\newcommand{\pdiff}[3]{\frac{\partial^{#3} #1}{\partial#2^{#3}}}
\newcommand{\abs}[1]{\left| #1 \right|}
\begin{document}
  \pagestyle{empty}
  \begin{minipage}{\linewidth}
    \centering
    \begin{tikzpicture}[very thick,font=\small]
      \node at (2,6) {\includegraphics[width=3.5cm]{logoudem}};
      \node at (9.5,6) {\includegraphics[width=9cm]{cbudem}};
      \node[fill=white,draw=white,inner sep=1mm] at (9.5,5.05) {\bf Permanencia con calidad, Acompa\~nar para exigir};
      \node[fill=white,draw=white,inner sep=1mm] at (7.5,4.2) {\Large\bf DEPARTAMENTO DE CIENCIAS B\'ASICAS};
      \draw (0,0) rectangle (18,3.5);
      \draw (0,2.5)--(18,2.5) (0,1.5)--(18,1.5) (15,4.2)--(18,4.2) node[below,pos=.5] {CALIFICACI\'ON} (15,2.5)--(15,7)--(18,7)--(18,3.5) (8.4,0)--(8.4,1.5) (15,0)--(15,1.5) (10,1.5)--(10,2.5);
      \node[right] at (0,3.2) {\bf Alumno:}; \node[right] at (15,3.2) {\bf Carn\'e:};
      \node[right] at (0,2.2) {\bf Asignatura:};
      \node at (6,1.95) {\asignatura};
      \node[right] at (10,2.2) {\bf Profesor:};
      \node at (15,1.95) {\profesor};
      \node[right] at (0,1.2) {\bf Examen:};
      \draw (3.8,.9) rectangle (4.4,1.3); \node[left] at (3.8,1.1) {Parcial:};
      \draw (3.8,.2) rectangle (4.4,.6); \node[left] at (3.8,.4) {Previa:};
      \draw (7.4,.9) rectangle (8,1.3); \node[left] at (7.4,1.1) {Final:};
      \draw (7.4,.2) rectangle (8,.6); \node[left] at (7.4,.4) {Habilitaci\'on:};
      \node at (4, .4) {X}; % Quiz
      \node[right] at (10,.5) {6 - 8}; % Número de grupo
      \node[right] at (10,1.) {12 de noviembre de 2015}; % Fecha de presentación
      \node[right] at (8.4,1.05) {\bf Fecha:}; \node[right] at (8.4,.45) {\bf Grupo:};
      \node[align=center,text width=3cm,font=\footnotesize] at (16.5,.75) {\centering\bf Use solo tinta\\y escriba claro};
    \end{tikzpicture}
  \end{minipage}
Indique \textbf{clara y explícitamente la respuesta final} de cada pregunta en \textbf{lapicero}, y \textbf{justifique} todas sus respuestas y procedimientos. Si algún elemento solicitado, en teoría no puede realizarse, indíquelo y justifique por que no se puede realizar lo solicitado. Respuestas sin justificación no cuentan.
  \begin{enumerate}[leftmargin=*,widest=9]
    \item Sea
    \[
    A = \begin{pmatrix}
    0.8 & 0.1\\ 0.1 & 1
    \end{pmatrix}
    \]
    \begin{enumerate}[label=\alph*]
    \item (\(1.0\)) ¿ Es \(A\) definida positiva?
    Se verifica el primer criterio.
    \[
    A^T = \begin{pmatrix}
    0.8 & 0.1\\ 0.1 & 1
    \end{pmatrix}^T = \begin{pmatrix}
    0.8 & 0.1\\ 0.1 & 1
    \end{pmatrix} = A
    \]
    La matriz \(A\) cumple con ser simétrica.
    El segundo criterio, puede validarse de dos formas. Una de ellas por medio de la pre y pos multiplicación por un vector arbitrario, y otra por el criterio de subdeterminantes (determinantes menores).
    Forma 1:
    \begin{eqnarray*}
    \vec{x}^T A \vec{x} & = & (x_1, x_2) \begin{pmatrix}
    0.8 & 0.1\\ 0.1 & 1
    \end{pmatrix} \begin{pmatrix}
    x_1\\ x_2
    \end{pmatrix} \\
    & = & (x_1, x_2) \begin{pmatrix}
    0.8x_1+0.1x_2\\ 0.1x_1+x_2
    \end{pmatrix} \\
    & = & 0.8x_1^2 +0.2x_1x_2 + x_2^2\\
    & = & 0.8(x_1+0.125x_2)^2+0.9875x_2^2 \quad \text{ Completación de cuadrado} \\
    & \geq & 0 \quad \text{ Por suma de cuadrados}
    \end{eqnarray*}
    Ahora, la única forma de ser cero, es el vector nulo, pero justamente este no se incluye en la definición, por lo que cumple con ser positivo para todo vector arbitrario y por ende es definida positiva.
    Forma 2:
    Se verifican los siguientes determinantes menores:
   \begin{eqnarray*}
   a_{11} &=& 0.8 > 0 \\
   \det(A) = \begin{vmatrix}
    0.8 & 0.1\\ 0.1 & 1
    \end{vmatrix} & = & 0.8(1) -0.1(0.1) = 0.79 > 0
   \end{eqnarray*}
   Como los determinantes menores son mayores que cero, cumple con ser definida positiva.
    \item (\(1.0\)) ¿Es \(A\) estrictamente diagonal dominante?
Se verifica que en la primera fila, \(|0.8| > |0.1|\), y en la segunda fila que \(|1| > |0.1|\), por lo que los elementos diagonales en valor absoluto son mayores a las sumas en valor absoluto de los demás elementos de la fila. Por ende, es estrictamente diagonal dominante.
    \end{enumerate}
    \item Sea
    \[
    B = \begin{pmatrix}
    -4 & 3 & 0\\ 2 & 8 & -5\\ 0 & 2 & 5
    \end{pmatrix}
    \]
    \begin{enumerate}[label=\alph*]
    \item (\(1.0\)) ¿Es singular \(B\)?
    Se observa que la matriz \(B\) es una matriz estrictamente diagonal dominante.
    \begin{eqnarray*}
    |-4| & > & |3|+|0|\\
    |8| & > & |2|+|-5|\\
    |5| & > & |0|+|2|
    \end{eqnarray*}
    Al ser estrictamente diagonal dominante, se asegura que la matriz es no singular (propiedad).
    \item (\(1.0\)) ¿Es \(\lambda = 0\) autovalor de \(B\)? (Recordar que \(\det(A) = \prod\limits_{i=1}^n \lambda_i\))
    Dado que \(B\) es una matriz no singular sabemos que su determinante es distinto de cero. Luego, ya que el determinante es igual a la productoria de los autovalores, bastaría un solo autovalor nulo para volver cero el determinante. Así, si el determinante de \(B\) es distinto de cero, implica que ningun autovalor es cero.
    \end{enumerate}
  \item (\(1.0\)) Demuestre que \( ||\vec{x}||_{\infty} \leq ||\vec{x}||_p\), para \(p \in \N\).
  Sea \(\vec{x} = (x_1, x_2, \ldots, x_n)\).
  Por definición, \( ||\vec{x}||_\infty = \max\limits_{1\leq i\leq n} |x_i|=|x_j|\) con \(j \in [1, n]\).
  Luego, si a este elemento, se suman los demás valores absolutos de los elementos del vector, cada uno elevado a una potencia natural, se obtiene la desigualdad de partida.
  \begin{eqnarray*}
  || \vec{x} ||_\infty &=& |x_j|\\
  || \vec{x} ||_\infty^p &=& |x_j|^p\\
  & \leq & |x_j|^p + \sum\limits_{\substack{i=0\\i\neq j}}^n |x_i|^p\\
  & \leq &  \sum\limits_{i=0}^n |x_i|^p\\
  (|| \vec{x} ||_\infty^p)^{1/p} & \leq & \left(\sum\limits_{i=0}^n |x_i|^p \right)^{1/p}\\
  || \vec{x} ||_\infty & \leq & || \vec{x} ||_p
  \end{eqnarray*}
  \end{enumerate}
\clearpage
\begin{center}
\textbf{Hoja de fórmulas}
\end{center}
{\large
\[
\begin{array}{cc}
h = \frac{b - a}{n} \qquad & \qquad
x_{i+2} = \frac{f(x_{i+1}){x_i}-f(x_{i}){x_{i+1}}}{f(x_{i+1}) - f(x_i)} \\
L_{n, i}(x) = \prod\limits_{\substack{j=0\\ i \neq j}}^n \frac{x - x_j}{x_i - x_j} \qquad & \qquad
P_n(x) = \sum\limits_{i = 0}^n f(x_i)L_{n,i}(x) \\
f\left[x_i, x_{i+1}\right] = \frac{f(x_{i+1})-f(x_i)}{x_{i+1}-x_i} \qquad & \qquad
f\left[ x_i, x_{i+1}, \ldots, x_{j-1}, x_j\right] = \frac{f\left[x_{i+1}, \ldots, x_{j-1}, x_j\right] - f\left[ x_i, x_{i+1}, \ldots, x_{j-1} \right]}{x_j - x_i} \\
z_{2i} = z_{2i+1} = x_i \qquad & \qquad
H_{2n+1} = f(x_0) + \sum\limits_{k=1}^{2n+1} f\left[x_0, \ldots, x_k\right] \prod\limits_{i = 0}^{k-1}(x-x_i)  \\
f^\prime(x_i) \approx \frac{f(x_{i+1}) - f(x_{i-1})}{2h} \qquad & \qquad
I = \frac{h}{3}\left( f(x_0) + f(x_n) + 2\sum\limits_{i=1}^{n/2-1}f(x_{2i}) + 4\sum\limits_{i=0}^{n/2-1}f(x_{2i+1}) \right) \\
f^\prime(x_i) = \frac{f(x_{i+1})-f(x_i)}{h} \qquad & \qquad
I = \frac{h}{2}\left( f(x_0) + f(x_n) + 2\sum\limits_{i = 1}^{n-1}f(x_i)\right) \\
\abs{\pdiff{f(t,y)}{y}{}} \leq L & \abs{f(t, y_1) -f(t, y_0)} \leq L\abs{y_1 - y_0}\\
W_{i+1} = y_i + W f(t_i,W_i) & W_{i+1} = W_i + h f(t_i,W_i) + \frac{h^2}{2} \left.\diff{f(t,W)}{t}{} \right|_{t_i, W_i} \\
k_1  =  h f(t_i,W_i) & k_2  =  h f(t_i+h/2,W_i + k_1/2) \\
k_4  =  h f(t_{i+1},W_i + k_3) & k_3  =  h f(t_i+h/2,W_i + k_2/2)\\
\vec{x}^TA\vec{x} > 0 & W_{i+1} = W_i + \frac{k_1+2k_2+2k_3+k_4}{6}\\
A = A^T & A = D - L - U\\
|a_{ii}| > \sum\limits_{\substack{j=0\\j\neq i}}^n |a_{ij}| & \vec{x}^{(k+1)} = T\vec{x}^{(k)} + \vec{c}\\
T_J = D^{-1}(L+U) & p(\lambda) = \det(A-\lambda \I) = 0\\
\vec{c}_J = D^{-1}\vec{b} & \rho(A) = \max\limits_{1\leq i\leq n}|\lambda_i|\\
T_G = (D-L)^{-1}U & \vec{c}_G = (D-L)^{-1}\vec{b}
\end{array}
\]
}
\end{document}
