\documentclass[12pt]{article}
\usepackage[letterpaper,margin={1.5cm}]{geometry}
\usepackage{amsmath, amssymb, amsfonts}
\usepackage[utf8]{inputenc}
\usepackage[T1]{fontenc}
\usepackage[spanish]{babel}
\usepackage{tikz}
\usepackage{graphicx,enumitem}
\usepackage{multicol}
\setlength{\marginparsep}{12pt} \setlength{\marginparwidth}{0pt} \setlength{\headsep}{.8cm} \setlength{\headheight}{15pt} \setlength{\labelwidth}{0mm} \setlength{\parindent}{0mm} \renewcommand{\baselinestretch}{1.15} \setlength{\fboxsep}{5pt} \setlength{\parskip}{3mm} \setlength{\arraycolsep}{2pt}
\renewcommand{\sin}{\operatorname{sen}}
\newcommand{\N}{\ensuremath{\mathbb{N}}}
\newcommand{\Z}{\ensuremath{\mathbb{Z}}}
\newcommand{\Q}{\ensuremath{\mathbb{Q}}}
\newcommand{\R}{\ensuremath{\mathbb{R}}}
\newcommand{\C}{\ensuremath{\mathbb{C}}}
\newcommand{\I}{\ensuremath{\mathbb{I}}}
\graphicspath{{../imagenes/}{imagenes/}{..}}
\allowdisplaybreaks{}
\raggedbottom{}
\setlength{\topskip}{0pt plus 2pt}
\newcommand{\profesor}{Edward Y. Villegas}
\newcommand{\asignatura}{M\'ETODOS NUM\'ERICOS}
\begin{document}
  \pagestyle{empty}
  \begin{minipage}{\linewidth}
    \centering
    \begin{tikzpicture}[very thick,font=\small]
      \node at (2,6) {\includegraphics[width=3.5cm]{logoudem}};
      \node at (9.5,6) {\includegraphics[width=9cm]{cbudem}};
      \node[fill=white,draw=white,inner sep=1mm] at (9.5,5.05) {\bf Permanencia con calidad, Acompa\~nar para exigir};
      \node[fill=white,draw=white,inner sep=1mm] at (7.5,4.2) {\Large\bf DEPARTAMENTO DE CIENCIAS B\'ASICAS};
      \draw (0,0) rectangle (18,3.5);
      \draw (0,2.5)--(18,2.5) (0,1.5)--(18,1.5) (15,4.2)--(18,4.2) node[below,pos=.5] {CALIFICACI\'ON} (15,2.5)--(15,7)--(18,7)--(18,3.5) (8.4,0)--(8.4,1.5) (15,0)--(15,1.5) (10,1.5)--(10,2.5);
      \node[right] at (0,3.2) {\bf Alumno:}; \node[right] at (15,3.2) {\bf Carn\'e:};
      \node[right] at (0,2.2) {\bf Asignatura:};
      \node at (6,1.95) {\asignatura};
      \node[right] at (10,2.2) {\bf Profesor:};
      \node at (15,1.95) {\profesor};
      \node[right] at (0,1.2) {\bf Examen:};
      \draw (3.8,.9) rectangle (4.4,1.3); \node[left] at (3.8,1.1) {Parcial:};
      \draw (3.8,.2) rectangle (4.4,.6); \node[left] at (3.8,.4) {Previa:};
      \draw (7.4,.9) rectangle (8,1.3); \node[left] at (7.4,1.1) {Final:};
      \draw (7.4,.2) rectangle (8,.6); \node[left] at (7.4,.4) {Habilitaci\'on:};
      \node at (4, .4) {X}; % Quiz
      \node[right] at (10,.5) {4}; % Número de grupo
      \node[right] at (10,1.) {18 de septiembre de 2015}; % Fecha de presentación
      \node[right] at (8.4,1.05) {\bf Fecha:}; \node[right] at (8.4,.45) {\bf Grupo:};
      \node[align=center,text width=3cm,font=\footnotesize] at (16.5,.75) {\centering\bf Use solo tinta\\y escriba claro};
    \end{tikzpicture}
  \end{minipage}
  Para la realización de este examen, puede usar calculadora científica o portátil.
  Reporte sus aproximaciones a mínimo \textbf{2 CIFRAS DECIMALES} con \textbf{REDONDEO SIMÉTRICO} en todas las operaciones y resultados.
  \begin{enumerate}[leftmargin=*,widest=9]
    \item \textbf{Interpolación} ($1.0$) : Sin expandir ni graficar, muestre si los siguientes polinomios son iguales o diferentes entre si.
    \begin{enumerate}[label=\alph*]
    \item \(P_1(x) = x^2\)
    \item \(P_2(x) = 2x(x-1)-x(x-2)\)
    \end{enumerate}
    Los polinomios $P_1(x)$ y $P_2(x)$ son de segundo grado. Acorde a la unicidad del polinomio interpolante de menor grado, sabemos que los polinomios de grado $n$ son determinados de forma única por $n+1$ puntos. De esta forma, basta con la evaluación arbitraria/ al azar de 3 puntos para determinar si son el mismo polinomio.
    \begin{center}
    \(    \begin{array}{|c|c|c|}
    \hline
    x_i & x_i^2 & 2x_i(x_i-1)-x_i(x_i-2)\\
    \hline
    0 & 0^2 = 0 & 2\cdot 0 \cdot (0-1)-0(0-2) = 0 \\
    1 & 1^2 = 1 & 2\cdot 1 \cdot (1-1)-1(1-2) = 1 \\
    2 & 2^2 = 4 & 2\cdot 2 \cdot (2-1)-2(2-2) = 4 \\
    \hline
    \end{array} \)
    \end{center}
    Como la evaluación de los 3 puntos coincide, se concluyo que los polinomios son iguales. Una cantidad de puntos inferior es coincidente con un infinito numero de polinomios y una cantidad mayor sin justificación no explica a partir de cuantas evaluaciones se puede considerar igual.
    \item \textbf{Métodos abiertos} ($2.0$) : Dado el polinomio \(P(x) = x^3 - 5x^2 + 17x -13\). Evalué la primera iteración de \textit{Newton Horner} con \(x_0 = 1.5\) (muestre explícitamente la evaluación de Horner).
    Sabemos que el método de Newton Horner posee la forma iterativa \(x_{i+1} = x_i - \frac{P(x)}{Q(x)}\) donde el numerador de la fracción es la evaluación del polinomio y el denominador es la evaluación de la derivada del polinomio, ambos con el método de Horner.
Así al realizar la primera aplicación del método de Horner se obtiene la evaluación del polinomio y tras la segunda aplicación se obtiene la derivada.
\begin{eqnarray*}
P(x) & = & x^3 - 5x^2 + 17x -13\\
b_3 & = & 1 \\
b_2 & = & (1)(1.5) - 5 = -3.5\\
b_1 & = & (-3.5)(1.5) + 17 = 11.75\\
b_0 & = & (11.75)(1.5) - 13 = 4.625 \\
Q(x) & = & x^2 - 3.5x + 11.75 \\
b^\prime_2 & = & 1 \\
b^\prime_1 & = & (1)(1.5) - 3.5 = -2\\
b^\prime_0 & = & (-2)(1.5) + 11.75 = 8.75\\
x_1 & = & 1.5 - \frac{4.625}{8.75} = 0.97143
\end{eqnarray*}
    \item \textbf{Punto fijo} ($2.0$): Dada la iteración de punto fijo \[ x_{n+1} = \frac{x_n + 1/x_n}{2},\] encuentre los valores sucesivos hasta \(x_4\) partiendo de \(x_0=1.5\).
 \centering   \( \begin{array}{|c|c|c|}
    \hline
    i & x_i & g(x_{i-1})\\
    \hline
    1 & 1.5 & \frac{1.5 + 1/1.5}{2} = 1.0833\\
    2 & 1.0833 & \frac{1.0833 + 1/1.0833}{2} = 1.0032\\
    3 & 1.0032 & \frac{1.0032 + 1/1.0032}{2} = 1.0000\\
    4 & 1.0000 & \frac{1.0000 + 1/1.0000}{2} = 1.0000\\
    \hline
    \end{array} \)
  \end{enumerate}
\end{document}
