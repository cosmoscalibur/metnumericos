\documentclass[12pt]{article}
\usepackage[letterpaper,margin={1.5cm}]{geometry}
\usepackage{amsmath, amssymb, amsfonts}
\usepackage[utf8]{inputenc}
\usepackage[T1]{fontenc}
\usepackage[spanish]{babel}
\usepackage{tikz}
\usepackage{graphicx,enumitem}
\usepackage{multicol}
\setlength{\marginparsep}{12pt} \setlength{\marginparwidth}{0pt} \setlength{\headsep}{.8cm} \setlength{\headheight}{15pt} \setlength{\labelwidth}{0mm} \setlength{\parindent}{0mm} \renewcommand{\baselinestretch}{1.15} \setlength{\fboxsep}{5pt} \setlength{\parskip}{3mm} \setlength{\arraycolsep}{2pt}
\renewcommand{\sin}{\operatorname{sen}}
\newcommand{\N}{\ensuremath{\mathbb{N}}}
\newcommand{\Z}{\ensuremath{\mathbb{Z}}}
\newcommand{\Q}{\ensuremath{\mathbb{Q}}}
\newcommand{\R}{\ensuremath{\mathbb{R}}}
\newcommand{\C}{\ensuremath{\mathbb{C}}}
\newcommand{\I}{\ensuremath{\mathbb{I}}}
\graphicspath{{../imagenes/}{imagenes/}{..}}
\allowdisplaybreaks{}
\raggedbottom{}
\usepackage{hyperref}
\setlength{\topskip}{0pt plus 2pt}
\newcommand{\profesor}{Edward Y. Villegas}
\newcommand{\asignatura}{M\'ETODOS NUM\'ERICOS}
\newcommand{\diff}[3]{\frac{d^{#3} #1}{d#2^{#3}}}
\newcommand{\pdiff}[3]{\frac{\partial^{#3} #1}{\partial#2^{#3}}}
\newcommand{\abs}[1]{\left| #1 \right|}
\begin{document}
  \pagestyle{empty}
  \begin{minipage}{\linewidth}
    \centering
    \begin{tikzpicture}[very thick,font=\small]
      \node at (2,6) {\includegraphics[width=3.5cm]{logoudem}};
      \node at (9.5,6) {\includegraphics[width=9cm]{cbudem}};
      \node[fill=white,draw=white,inner sep=1mm] at (9.5,5.05) {\bf Permanencia con calidad, Acompa\~nar para exigir};
      \node[fill=white,draw=white,inner sep=1mm] at (7.5,4.2) {\Large\bf DEPARTAMENTO DE CIENCIAS B\'ASICAS};
      \draw (0,0) rectangle (18,3.5);
      \draw (0,2.5)--(18,2.5) (0,1.5)--(18,1.5) (15,4.2)--(18,4.2) node[below,pos=.5] {CALIFICACI\'ON} (15,2.5)--(15,7)--(18,7)--(18,3.5) (8.4,0)--(8.4,1.5) (15,0)--(15,1.5) (10,1.5)--(10,2.5);
      \node[right] at (0,3.2) {\bf Alumno:}; \node[right] at (15,3.2) {\bf Carn\'e:};
      \node[right] at (0,2.2) {\bf Asignatura:};
      \node at (6,1.95) {\asignatura};
      \node[right] at (10,2.2) {\bf Profesor:};
      \node at (15,1.95) {\profesor};
      \node[right] at (0,1.2) {\bf Examen:};
      \draw (3.8,.9) rectangle (4.4,1.3); \node[left] at (3.8,1.1) {Parcial:};
      \draw (3.8,.2) rectangle (4.4,.6); \node[left] at (3.8,.4) {Previa:};
      \draw (7.4,.9) rectangle (8,1.3); \node[left] at (7.4,1.1) {Final:};
      \draw (7.4,.2) rectangle (8,.6); \node[left] at (7.4,.4) {Habilitaci\'on:};
      \node[right] at (10,.5) {8}; % Número de grupo
      \node[right] at (10,1.) {4 de septiembre de 2016}; % Fecha de presentación
      \node[right] at (8.4,1.05) {\bf Fecha:}; \node[right] at (8.4,.45) {\bf Grupo:};
      \node[align=center,text width=3cm,font=\footnotesize] at (16.5,.75) {\centering\bf PDF\\o Notebook};
    \end{tikzpicture}
  \end{minipage}
Todo valor reportado debe aproximarse a 5 cifras significativas con redondeo simétrico (no es necesario en enteros y valores dados en enunciado) salvo que se indique lo contrario en el enunciado, y con coma decimal (en la entrada y salida de sus códigos será el separador por defecto del lenguaje).
Si algún elemento solicitado en teoría no puede realizarse, justifique por que no se puede realizar lo solicitado como respuesta (su código debe estar en capacidad de detectar posibles casos fallidos).
Para las demás indicaciones remítase al \href{https://www.dropbox.com/s/noko8eysm8une33/CondicionesEntrega.pdf?dl=0}{documento de parámetros de entrega de documentos digitales}.
\vspace{-.5cm}
  \begin{enumerate}[leftmargin=*,widest=9]
    \item Elabore una función que permita evaluar un polinomio por el algoritmo de Horner, al igual que su derivada. La implementación del algoritmo de Horner debe ser propia y no una rutina de un paquete del lenguaje que use.
   \begin{enumerate}[label=\alph*]
    \item Argumento 1: Debe recibir el polinomio en forma de arreglo o lista, siendo el primer elemento de la lista de izquierda a derecha el coeficiente de mayor grado. Ejemplo: \(4x^2-1\) se expresaría como \verb+[4 0 -1]+.
    \item Argumento 2: Recibe un número flotante para su evaluación.
    \item Argumento 3: \verb-f- evaluá el polinomio, \verb-d- evaluá el polinomio y su derivada (este último debe ser el comportamiento por defecto).
    \item Salida: Devuelve un número si se usa el argumento \verb-f-, y una lista de números si usa el argumento \verb-d-, siendo el primer número de izquierda a derecha la evaluación del polinomio.
    \end{enumerate}
Ilustre el correcto funcionamiento de la función probando con los casos:
\begin{enumerate}[label=\roman*]
\item  \(4x^2-1\) con \(x=1.6\) y evaluando solo el polinomio.
\item \(5.5x^3 - 4x^2 + 1\) con \(x=4\) y evaluando el polinomio y su derivada.
 \end{enumerate}
    \item Elabore una función para el uso del método de Newton-Horner de búsqueda de raíces para polinomios.
    \begin{enumerate}[label=\alph*]
    \item Argumentos: Polinomio, Valor inicial, Tolerancia, máximo de iteraciones e indicador para generar tabla.
    \item La evaluación del polinomio debe recurrir a la función ya definida en el punto anterior.
	\item El indicador para generar tabla es un caracter que indica si genera la tabla de las iteraciones. \verb-y- para generar tabla y \verb-n- para no generarla (este último debe ser el comportamiento por defecto). La tabla debe tener las columnas de número de iteración, aproximación de la raíz, evaluación del polinomio, evaluación de la derivada, nueva aproximación y error absoluto.
\end{enumerate}
Para la prueba, resuelva el siguiente problema aplicado: La sumatoria del flujo de caja durante un periodo de tiempo medido en años para una inversión dada, puede ser modelada por \(f(x) = {(x-5.5)}^3 + x - 20\). Se denomina punto de equilibrio al tiempo para el cual la inversión realizada se recupera, siendo esto equivalente a que la sumatoria del flujo de caja sea cero. Use unidades si corresponde al reportar la respuesta final. ¿Al cabo de cuanto tiempo se logra el punto de equilibrio?
\end{enumerate}
\end{document}
