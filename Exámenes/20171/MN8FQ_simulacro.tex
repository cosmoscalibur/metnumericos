\documentclass[12pt]{article}
\usepackage[letterpaper,margin={1.5cm}]{geometry}
\usepackage{amsmath, amssymb, amsfonts}
\usepackage[utf8]{inputenc}
\usepackage[T1]{fontenc}
\usepackage[spanish]{babel}
\usepackage{tikz}
\usepackage{graphicx,enumitem}
\usepackage{multicol}
\usepackage{hyperref}
\setlength{\marginparsep}{12pt} \setlength{\marginparwidth}{0pt} \setlength{\headsep}{.8cm} \setlength{\headheight}{15pt} \setlength{\labelwidth}{0mm} \setlength{\parindent}{0mm} \renewcommand{\baselinestretch}{1.15} \setlength{\fboxsep}{5pt} \setlength{\parskip}{3mm} \setlength{\arraycolsep}{2pt}
\renewcommand{\sin}{\operatorname{sen}}
\newcommand{\N}{\ensuremath{\mathbb{N}}}
\newcommand{\Z}{\ensuremath{\mathbb{Z}}}
\newcommand{\Q}{\ensuremath{\mathbb{Q}}}
\newcommand{\R}{\ensuremath{\mathbb{R}}}
\newcommand{\C}{\ensuremath{\mathbb{C}}}
\newcommand{\I}{\ensuremath{\mathbb{I}}}
\graphicspath{{../imagenes/}{imagenes/}{..}}
\a\allowdisplaybreaks{}
\r\raggedbottom{}
\setlength{\topskip}{0pt plus 2pt}
\newcommand{\profesor}{Simulacro}
\newcommand{\asignatura}{MÉTODOS NUMÉRICOS}
\newcommand{\diff}[3]{\frac{d^{#3} #1}{d#2^{#3}}}
\newcommand{\pdiff}[3]{\frac{\partial^{#3} #1}{\partial#2^{#3}}}
\newcommand{\abs}[1]{\left| #1 \right|}
\begin{document}
  \pagestyle{empty}
  \begin{minipage}{\linewidth}
    \centering
    \begin{tikzpicture}[very thick,font=\small]
      \node at (2,6) {\includegraphics[width=3.5cm]{logoudem}};
      \node at (9.5,6) {\includegraphics[width=9cm]{cbudem}};
      \node[fill=white,draw=white,inner sep=1mm] at (9.5,5.05) {\bf Permanencia con calidad, Acompañar para exigir};
      \node[fill=white,draw=white,inner sep=1mm] at (7.5,4.2) {\Large\bf DEPARTAMENTO DE CIENCIAS BÁSICAS};
      \draw (0,0) rectangle (18,3.5);
      \draw (0,2.5)--(18,2.5) (0,1.5)--(18,1.5) (15,4.2)--(18,4.2) node[below,pos=.5] {CALIFICACIÓN} (15,2.5)--(15,7)--(18,7)--(18,3.5) (8.4,0)--(8.4,1.5) (15,0)--(15,1.5) (10,1.5)--(10,2.5);
      \node[right] at (0,3.2) {\bf Alumno:}; \node[right] at (15,3.2) {\bf Carn\'e:};
      \node[right] at (0,2.2) {\bf Asignatura:};
      \node at (6,1.95) {\asignatura};
      \node[right] at (10,2.2) {\bf Profesor:};
      \node at (15,1.95) {\profesor};
      \node[right] at (0,1.2) {\bf Examen:};
      \draw (3.8,.9) rectangle (4.4,1.3); \node[left] at (3.8,1.1) {Parcial:};
      \draw (3.8,.2) rectangle (4.4,.6); \node[left] at (3.8,.4) {Previa:};
      \draw (7.4,.9) rectangle (8,1.3); \node[left] at (7.4,1.1) {Final:};
      \draw (7.4,.2) rectangle (8,.6); \node[left] at (7.4,.4) {Habilitación:};
      \node at (7.6, 1.1) {X}; % Final
      \node[right] at (10,.5) {}; % Número de grupo
      \node[right] at (10,1.) {24 de mayo de 2017}; % Fecha de presentación
      \node[right] at (8.4,1.05) {\bf Fecha:}; \node[right] at (8.4,.45) {\bf Grupo:};
      \node[align=center,text width=3cm,font=\footnotesize] at (16.5,.75) {\centering\bf Use solo tinta\\y escriba claro};
    \end{tikzpicture}
  \end{minipage}
Se permite usar calculadora de cualquier tipo mas no el uso de portátil o celulares en el examen, y tendrá como apoyo una hoja de formulas al final.
Todo valor reportado debe aproximarse a 5 cifras significativas con redondeo simétrico (no es necesario en enteros y valores dados en enunciado) salvo que se indique lo contrario en el enunciado, con separador decimal consistente y unidades si lo requiere.
En caso de reclamación solo cuenta lo que este en lapicero. Si algún elemento solicitado en teoría no puede realizarse, justifique por que no se puede realizar lo solicitado como respuesta. Toda respuesta debe estar adecuadamente justificada y/o con procedimiento, y resuelva las preguntas en orden.
\vspace{-.5cm}
  \begin{enumerate}[leftmargin=*,widest=9]
%% Punto 1
    \item Una compañía ajusta el modelo a los datos de ventas mensuales, encontrando la relación
    \( S(t)= \frac{t}{4}+1.8+0.5\sin\left( \frac{\pi t}{6} \right),
    \)
    donde \(S\) son las ventas (en miles) y \(t\) es el tiempo en meses. Se desea calcular el promedio de ventas mensuales del primer semestre, recordando que el promedio se calcula como:
    \(
    \bar{S} = \int\limits_{t_i}^{t_f}S(t)dt /(t_f - t_i)
    \)
    \begin{enumerate}[label=\alph*]
    \item Obtenga una aproximación al promedio de venta del primer semestre (\(t\) de 0 a 6 meses) con una regla compuesta de su preferencia.
    \item Calcule la cota de error máxima para su aproximación de la integral con la regla compuesta usada.
    \item Determine el número de cifras significativas de la aproximación y reescriba el promedio de venta acorde a esto.
    \end{enumerate}
%% Punto 2
    \item La biomasa \(b(t)\) (en toneladas) de un ecosistema dado con biomasa inicial de 7 toneladas evoluciona anualmente acorde al modelo siguiente:
    \[
\diff{b(t)}{t}{} = 3.5 - 0.019b(t).
    \]
    \begin{enumerate}[label=\alph*]
    \item ¿Cumple con la condición de solución única asegurada el problema de valor inicial?
    \item Aproxime con el método de Runge Kutta 4 la biomasa trascurrido un año usando dos pasos temporales.
\end{enumerate}
        % Punto 3
   \item Dado el sistema de ecuaciones lineales
   \begin{align*}
   3x_1 - 0.1x_2 - 0.2x_3 & = 7.85 \\
   0.1x_1 + 7x_2 - 0.3x_3 & = -19.3 \\
   0.3x_1 - 0.2x_2 + 10x_3 & = 71.4
   \end{align*}
   \begin{enumerate}[label=\alph*]
   \item Determine la matriz \(T\) y el vector \(\vec{c}\) para el método de su preferencia, para el sistema \(A\vec{x}=\vec{b}\) equivalente al sistema de ecuaciones.
   \item Valide que el método seleccionado será convergente a la solución única del problema.
   \item Usando como aproximación inicial el vector \( \begin{pmatrix} 2, -2, \ 5 \end{pmatrix}^T \), muestre la solución aproximada tras una iteración.
   \item Calcule el error absoluto de la aproximación usando la norma 3 con respecto a la solución exacta. La solución exacta es el vector
   \( \begin{pmatrix} 3, -2.5, 7 \end{pmatrix}^T \).
\end{enumerate}
   \end{enumerate}
\end{document}
