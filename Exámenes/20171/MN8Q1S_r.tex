\documentclass[12pt]{article}

\usepackage[letterpaper,margin={1.5cm}]{geometry}
\usepackage{amsmath, amssymb, amsfonts}

\usepackage[utf8]{inputenc}
\usepackage[T1]{fontenc}
\usepackage[spanish]{babel}
\usepackage{tikz}
\usepackage{graphicx,enumitem}
\usepackage{multicol}
\usepackage{hyperref}

\setlength{\marginparsep}{12pt} \setlength{\marginparwidth}{0pt} \setlength{\headsep}{.8cm} \setlength{\headheight}{15pt} \setlength{\labelwidth}{0mm} \setlength{\parindent}{0mm} \renewcommand{\baselinestretch}{1.15} \setlength{\fboxsep}{5pt} \setlength{\parskip}{3mm} \setlength{\arraycolsep}{2pt}

\renewcommand{\sin}{\operatorname{sen}}
\newcommand{\N}{\ensuremath{\mathbb{N}}}
\newcommand{\Z}{\ensuremath{\mathbb{Z}}}
\newcommand{\Q}{\ensuremath{\mathbb{Q}}}
\newcommand{\R}{\ensuremath{\mathbb{R}}}
\newcommand{\C}{\ensuremath{\mathbb{C}}}
\newcommand{\I}{\ensuremath{\mathbb{I}}}

\graphicspath{{../imagenes/}}

\allowdisplaybreaks

\raggedbottom
\setlength{\topskip}{0pt plus 2pt}



\newcommand{\profesor}{Edward Y. Villegas}
\newcommand{\asignatura}{M\'ETODOS NUM\'ERICOS}



\newcommand{\diff}[3]{\frac{d^{#3} #1}{d#2^{#3}}}
\newcommand{\pdiff}[3]{\frac{\partial^{#3} #1}{\partial #2^{#3}}}
\newcommand{\abs}[1]{\left| #1 \right|}

\begin{document}
  \pagestyle{empty}
  \begin{minipage}{\linewidth}
    \centering
    \begin{tikzpicture}[very thick,font=\small]

      \node at (2,6) {\includegraphics[width=3.5cm]{logoudem}};
      \node at (9.5,6) {\includegraphics[width=9cm]{cbudem}};
      \node[fill=white,draw=white,inner sep=1mm] at (9.5,5.05) {\bf Permanencia con calidad, Acompa\~nar para exigir};
      \node[fill=white,draw=white,inner sep=1mm] at (7.5,4.2) {\Large\bf DEPARTAMENTO DE CIENCIAS B\'ASICAS};
      \draw (0,0) rectangle (18,3.5);
      \draw (0,2.5)--(18,2.5) (0,1.5)--(18,1.5) (15,4.2)--(18,4.2) node[below,pos=.5] {CALIFICACI\'ON} (15,2.5)--(15,7)--(18,7)--(18,3.5) (8.4,0)--(8.4,1.5) (15,0)--(15,1.5) (10,1.5)--(10,2.5);
      \node[right] at (0,3.2) {\bf Alumno:}; \node[right] at (15,3.2) {\bf Carn\'e:};
      \node[right] at (0,2.2) {\bf Asignatura:};
      \node at (6,1.95) {\asignatura};
      \node[right] at (10,2.2) {\bf Profesor:};
      \node at (15,1.95) {\profesor};
      \node[right] at (0,1.2) {\bf Examen:};
      \draw (3.8,.9) rectangle (4.4,1.3); \node[left] at (3.8,1.1) {Parcial:};
      \draw (3.8,.2) rectangle (4.4,.6); \node[left] at (3.8,.4) {Previa:};
      \draw (7.4,.9) rectangle (8,1.3); \node[left] at (7.4,1.1) {Final:};
      \draw (7.4,.2) rectangle (8,.6); \node[left] at (7.4,.4) {Habilitaci\'on:};

       \node at (4, .4) {X}; % Quiz


      \node[right] at (10,.5) {8}; % Número de grupo
      \node[right] at (10,1.) {17 de mayo de 2017}; % Fecha de presentación
      \node[right] at (8.4,1.05) {\bf Fecha:}; \node[right] at (8.4,.45) {\bf Grupo:};
      \node[align=center,text width=3cm,font=\footnotesize] at (16.5,.75) {\centering\bf Use solo tinta\\y escriba claro};
    \end{tikzpicture}
  \end{minipage}

  Se permite usar calculadora de cualquier tipo mas no el uso de portátil o celulares en el examen, y puede disponer de sus apuntes de clase.
  Todo valor reportado debe aproximarse a 5 cifras significativas con redondeo simétrico (no es necesario en enteros y valores dados en enunciado) salvo que se indique lo contrario en el enunciado, con separador decimal consistente y unidades si lo requiere.
  En caso de reclamación solo cuenta lo que este en lapicero. Si algún elemento solicitado en teoría no puede realizarse, justifique por que no se puede realizar lo solicitado como respuesta. Toda respuesta debe estar adecuadamente justificada y/o con procedimiento, y resuelva las preguntas en orden.

\vspace{-.5cm}
  \begin{enumerate}[leftmargin=*,widest=9]


    \item Dado el siguiente pseudocódigo

\textbf{Inicio}
\(a\gets 5\)\\
\textbf{Mientras} \(a \geq 0\)\\
\hspace*{1cm}\(a \gets a - \exp(a)\)\\
\hspace*{1cm}\(b \gets \exp(a) \)\\
\textbf{Fin Mientras}\\
\textbf{Fin}

\vspace{-0.5cm}

    \begin{enumerate}[label=\alph*]
    \item (\(0.7\)) ¿Cual es el valor de la variable \(b\) la primera vez que es asignada?
\vspace{1.5cm}
    \item (\(0.3\)) ¿Cuantas veces se ejecutará el ciclo indicado en el pseudocódigo? Justifique la respuesta.
\vspace{1.5cm}
    \end{enumerate}


    \item Un cultivo de bacterias con una población inicial (\(P_0\)) de 100 individuos aumenta su población acorde a la ley exponencial \(P = P_0\exp(rt)\) donde su tasa de crecimiento por hora es de \(r=0.0263\). Use unidades en la respuesta final si es requerido por la variable.

    \begin{enumerate}[label=\alph*]
    \item (\(0.7\)) Determine una aproximación numérica a máximo 3 iteraciones del tiempo necesario para que la población de bacterias sea de 1500 individuos. Escoja el método y parámetros requeridos justificando el motivo de la elección.
\vspace*{5cm}
    \item (\(0.5\)) Sabiendo que una buena aproximación es \(10.297\) años, determine el error absoluto de su aproximación.
\vspace{1cm}
    \item (\(0.7\)) Con la información del literal anterior, determine el número de cifras significativas de su aproximación de manera formal.
\vspace{3cm}
\item (\(0.5\)) Reporte su resultado aproximado acorde a las cifras significativas que determino en el literal anterior.
\vspace{0.5cm}
\end{enumerate}

   \item Sea \(f(x) = \ln(x)\), se desea conocer una aproximación de primer orden de la función para \(x=0.1\) usando como punto central \(x_0=0.0\).

   \begin{enumerate}[label=\alph*]
    \item (\(0.8\)) ¿Cumple el teorema de aproximación de Taylor el problema planteado?
\vspace{3cm}

\item (\(0.8\)) Realice la aproximación indicada.
\vspace{3cm}
  \end{enumerate}
\end{enumerate}

\end{document}
