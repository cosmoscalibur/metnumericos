\documentclass[12pt]{article}
\usepackage[letterpaper,margin={1.5cm}]{geometry}
\usepackage{amsmath, amssymb, amsfonts}
\usepackage[utf8]{inputenc}
\usepackage[T1]{fontenc}
\usepackage[spanish]{babel}
\usepackage{tikz}
\usepackage{graphicx,enumitem}
\usepackage{multicol}
\setlength{\marginparsep}{12pt} \setlength{\marginparwidth}{0pt} \setlength{\headsep}{.8cm} \setlength{\headheight}{15pt} \setlength{\labelwidth}{0mm} \setlength{\parindent}{0mm} \renewcommand{\baselinestretch}{1.15} \setlength{\fboxsep}{5pt} \setlength{\parskip}{3mm} \setlength{\arraycolsep}{2pt}
\renewcommand{\sin}{\operatorname{sen}}
\newcommand{\N}{\ensuremath{\mathbb{N}}}
\newcommand{\Z}{\ensuremath{\mathbb{Z}}}
\newcommand{\Q}{\ensuremath{\mathbb{Q}}}
\newcommand{\R}{\ensuremath{\mathbb{R}}}
\newcommand{\C}{\ensuremath{\mathbb{C}}}
\newcommand{\I}{\ensuremath{\mathbb{I}}}
\graphicspath{{../imagenes/}{imagenes/}{..}}
\allowdisplaybreaks{}
\raggedbottom{}
\setlength{\topskip}{0pt plus 2pt}
\newcommand{\profesor}{Edward Y. Villegas}
\newcommand{\asignatura}{MÉTODOS NUMÉRICOS}
\newcommand{\diff}[3]{\frac{d^{#3} #1}{d#2^{#3}}}
\newcommand{\pdiff}[3]{\frac{\partial^{#3} #1}{\partial#2^{#3}}}
\newcommand{\abs}[1]{\left| #1 \right|}
\begin{document}
  \pagestyle{empty}
  \begin{minipage}{\linewidth}
    \centering
    \begin{tikzpicture}[very thick,font=\small]
      \node at (2,6) {\includegraphics[width=3.5cm]{logoudem}};
      \node at (9.5,6) {\includegraphics[width=9cm]{cbudem}};
      \node[fill=white,draw=white,inner sep=1mm] at (9.5,5.05) {\bf Permanencia con calidad, Acompañar para exigir};
      \node[fill=white,draw=white,inner sep=1mm] at (7.5,4.2) {\Large\bf DEPARTAMENTO DE CIENCIAS BÁSICAS};
      \draw (0,0) rectangle (18,3.5);
      \draw (0,2.5)--(18,2.5) (0,1.5)--(18,1.5) (15,4.2)--(18,4.2) node[below,pos=.5] {CALIFICACIÓN} (15,2.5)--(15,7)--(18,7)--(18,3.5) (8.4,0)--(8.4,1.5) (15,0)--(15,1.5) (10,1.5)--(10,2.5);
      \node[right] at (0,3.2) {\bf Alumno:}; \node[right] at (15,3.2) {\bf Carné:};
      \node[right] at (0,2.2) {\bf Asignatura:};
      \node at (6,1.95) {\asignatura};
      \node[right] at (10,2.2) {\bf Profesor:};
      \node at (15,1.95) {\profesor};
      \node[right] at (0,1.2) {\bf Examen:};
      \draw (3.8,.9) rectangle (4.4,1.3); \node[left] at (3.8,1.1) {Parcial:};
      \draw (3.8,.2) rectangle (4.4,.6); \node[left] at (3.8,.4) {Previa:};
      \draw (7.4,.9) rectangle (8,1.3); \node[left] at (7.4,1.1) {Final:};
      \draw (7.4,.2) rectangle (8,.6); \node[left] at (7.4,.4) {Habilitación:};
       \node at (4, .4) {X}; % Quiz
      \node[right] at (10,.5) {8}; % Número de grupo
      \node[right] at (10,1.) {14 de abril de 2017}; % Fecha de presentación
      \node[right] at (8.4,1.05) {\bf Fecha:}; \node[right] at (8.4,.45) {\bf Grupo:};
      \node[align=center,text width=3cm,font=\footnotesize] at (16.5,.75) {\centering\bf Use solo tinta\\y escriba claro};
    \end{tikzpicture}
  \end{minipage}
  Se permite usar calculadora de cualquier tipo mas no el uso de portátil o celulares en el examen, y puede disponer de sus apuntes de clase.
  Todo valor reportado debe aproximarse a 5 cifras significativas con redondeo simétrico (no es necesario en enteros y valores dados en enunciado) salvo que se indique lo contrario en el enunciado, con separador decimal consistente y unidades si lo requiere.
  En caso de reclamación solo cuenta lo que este en lapicero. Si algún elemento solicitado en teoría no puede realizarse, justifique por que no se puede realizar lo solicitado como respuesta. Toda respuesta debe estar adecuadamente justificada y/o con procedimiento, y resuelva las preguntas en orden.
\vspace{-.5cm}
  \begin{enumerate}[leftmargin=*,widest=9]
    \item Se desea realizar la integral de la función \(\exp(-x)\) en el intervalo cerrado \(\left[0, 1\right]\).
    \begin{enumerate}[label=\alph*]
    \item (\(0.6\)) Aproxime el resultado mediante la aplicación de un método con la forma simple.
		\textbf{R:} Dado que no se especifico un criterio, se procede a ejemplificar la solución con el método del trapecio.
    \[
    \int_{0}^{1}\exp(-x)dx \approx \frac{1-0}{2}(\exp(-(1)) + \exp(-(0)))=\frac{0.36788 + 1}{2}=0.68394
    \]
    \item (\(1.0\)) Aproxime el resultado mediante la aplicación de la forma compuesta del método anterior con al menos 2 segmentos.
		\textbf{R:} La forma compuesta de trapecio es posible usarla para cualquier número de intervalos. El número de intervalos que se usará son 2 y el tamaño de paso se obtiene como
    \[
    h = \frac{1-0}{2} = 0.50000.
    \]
    Así, la aplicación del método compuesto resultaría de considerar los puntos \( x=\lbrace 0, 0.50000, 1 \rbrace \), y por ende
    \[
    \int_{0}^{1}\exp(-x)dx \approx \frac{0.50000}{2}(\exp(-0)+2\exp(-0.50000)+\exp(-1))=0.64524
    \]
    \item (\(1.0\)) Calcule la cota de error máxima para su aproximación de la integral con la regla compuesta usada.
		\textbf{R:} Para la cota de error del trapecio requerimos la segunda derivada de la función que se integraba y maximizar su valor en el intervalo dado por los limites (multiplicado por una constante que depende del tamaño del intervalo y número de intervalos).
    \begin{eqnarray*}
e_a &\leq& \max_{0\leq\mu\leq 1}\left\vert -\frac{(1-0)0.50000^2}{12} \frac{d^2\exp(-\mu)}{d\mu} \right\vert \\
& \leq& \frac{(1-0)0.50000^2}{12}\max_{0\leq\mu\leq 1}\exp(-\mu) \\
&\leq & 0.020833 \exp(0) = 0.020833
    \end{eqnarray*}
    Así la cota de error máximo absoluto es de \(0.020833\).
    \item (\(0.8\)) Con base a la cota de error obtenida, reescriba la aproximación de la forma compuesta con las cifras significativas adecuadas.
    \textbf{R:} Al desconocer el valor verdadero de la integral, se obtiene el valor del error relativo con base a la aproximación como valor verdadero y el error absoluto a partir de la cota.
\begin{eqnarray*}
e_r = \frac{0.020833}{0.64524} = 0.032287\\
0.032287  = 0.32287\cdot 10^{-1} \leq 0.5 \cdot 10^{-n+1}\\
-1 = -n + 1\\
n = 2
\end{eqnarray*}
El número de cifras significativas de la aproximación es 2, y por ende el resultado de la integral debe reescribirse como \(0.65\).
    \end{enumerate}
    \item Dada la superficie descrita por la semiesfera del hemisferio positivo \(z=\sqrt{1 - x^2 - y^2}\), se desea determinar la aproximación del volumen bajo la superficie en el dominio dado por el cuadrilátero irregular cuyos vértices son
    \(
    P = \left\lbrace (-0.3, -0.1), (0.35, -0.25), (0.3, 0.2), (-0.27, 0.3)\right\rbrace
    \)
    con el método de integración sobre elementos finitos.
    \begin{enumerate}[label=\alph*]
    \item (\(0.6\)) Especifique una matriz de nodos y una matriz de elementos para la solución de la integral doble usando un mallado de 2 triángulos.
	   \textbf{R:}
Inicialmente se requiere construir la matriz de nodos, la cual especifica las coordenadas de los puntos de la malla.
     \[
     \begin{array}{|c|c|c|}
     \hline
     i & x & y\\
     \hline
     0 & -0.3 & -0.1\\
     1 & 0.35 & -0.25\\
     2 & 0.3 & 0.2\\
     3 & -0.27 & 0.3\\
     \hline
     \end{array}
     \]
Con esta matriz conocida, es posible referenciar la matriz de elementos o adyacencia respecto a esta.
\[
\begin{array}{|c|c|c|c|}
\hline
j & P_0 & P_1 & P_2 \\
\hline
0 & 0 & 1 & 2\\
1 & 2 & 3 & 0\\
\hline
\end{array}
\]
Es posible tener los mismos elementos con la asignación de nodos diferente siempre que sea en sentido antihorario. Igualmente existe otra configuración de triángulos también valida.
    \item (\(1.0\)) Aproxime el valor númerico de la integral con base al mallado del dominio que realizo.
	  \textbf{R:}
Triangulo 0:
\begin{eqnarray*}
a_0 = \sqrt{(x_{01} - x_{00})^2 + (y_{01}-y_{00})^2} = 0.66708\\
b_0 = \sqrt{(x_{02} - x_{01})^2 + (y_{02}-y_{01})^2} = 0.45277\\
c_0 = \sqrt{(x_{00} - x_{02})^2 + (y_{00}-y_{02})^2} = 0.67082\\
s_0 = \frac{a_0+b_0+c_0}{2} = 0.89534\\
A_0 = \sqrt{s_0(s_0 - a_0)(s_0-b_0)(s_0-c_0)} = 0.14250\\
h_0 = \frac{z(x_{00},y_{00})+z(x_{01},y_{01})+z(x_{02},y_{02})}{3} 0.92806 \\
V_0 = h_0 A_0 = 0.13225
\end{eqnarray*}
De manera equivalente desarrollamos para el otro triangulo.
Triangulo 1:
\begin{eqnarray*}
a_1 = \sqrt{(x_{11} - x_{10})^2 + (y_{11}-y_{10})^2} = 0.57871\\
b_1 = \sqrt{(x_{12} - x_{11})^2 + (y_{12}-y_{11})^2} = 0.40112\\
c_1 = \sqrt{(x_{10} - x_{12})^2 + (y_{10}-y_{12})^2} = 0.67082\\
s_1 = \frac{a_1+b_1+c_1}{2} = 0.82532\\
A_1 = \sqrt{s_1(s_1 - a_1)(s_1-b_1)(s_1-c_1)} = 0.11550\\
h_1 = \frac{z(x_{10},y_{10})+z(x_{11},y_{11})+z(x_{12},y_{12})}{3} 0.93212 \\
V_1 = h_1 A_1 = 0.10766
\end{eqnarray*}
Total: \(\iint_D (1-x^2-y^2)dxdx \approx V_0 + V1 = 0.23991\)
    \end{enumerate}
\end{enumerate}
\end{document}
