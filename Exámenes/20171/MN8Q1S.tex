\documentclass[12pt]{article}
\usepackage[letterpaper,margin={1.5cm}]{geometry}
\usepackage{amsmath, amssymb, amsfonts}
\usepackage[utf8]{inputenc}
\usepackage[T1]{fontenc}
\usepackage[spanish]{babel}
\usepackage{tikz}
\usepackage{graphicx,enumitem}
\usepackage{multicol}
\setlength{\marginparsep}{12pt} \setlength{\marginparwidth}{0pt} \setlength{\headsep}{.8cm} \setlength{\headheight}{15pt} \setlength{\labelwidth}{0mm} \setlength{\parindent}{0mm} \renewcommand{\baselinestretch}{1.15} \setlength{\fboxsep}{5pt} \setlength{\parskip}{3mm} \setlength{\arraycolsep}{2pt}
\renewcommand{\sin}{\operatorname{sen}}
\newcommand{\N}{\ensuremath{\mathbb{N}}}
\newcommand{\Z}{\ensuremath{\mathbb{Z}}}
\newcommand{\Q}{\ensuremath{\mathbb{Q}}}
\newcommand{\R}{\ensuremath{\mathbb{R}}}
\newcommand{\C}{\ensuremath{\mathbb{C}}}
\newcommand{\I}{\ensuremath{\mathbb{I}}}
\graphicspath{{../imagenes/}{imagenes/}{..}}
\allowdisplaybreaks{}
\raggedbottom{}
\setlength{\topskip}{0pt plus 2pt}
\newcommand{\profesor}{Edward Y. Villegas}
\newcommand{\asignatura}{M\'ETODOS NUM\'ERICOS}
\newcommand{\diff}[3]{\frac{d^{#3} #1}{d#2^{#3}}}
\newcommand{\pdiff}[3]{\frac{\partial^{#3} #1}{\partial#2^{#3}}}
\newcommand{\abs}[1]{\left| #1 \right|}
\begin{document}
  \pagestyle{empty}
  \begin{minipage}{\linewidth}
    \centering
    \begin{tikzpicture}[very thick,font=\small]
      \node at (2,6) {\includegraphics[width=3.5cm]{logoudem}};
      \node at (9.5,6) {\includegraphics[width=9cm]{cbudem}};
      \node[fill=white,draw=white,inner sep=1mm] at (9.5,5.05) {\bf Permanencia con calidad, Acompa\~nar para exigir};
      \node[fill=white,draw=white,inner sep=1mm] at (7.5,4.2) {\Large\bf DEPARTAMENTO DE CIENCIAS B\'ASICAS};
      \draw (0,0) rectangle (18,3.5);
      \draw (0,2.5)--(18,2.5) (0,1.5)--(18,1.5) (15,4.2)--(18,4.2) node[below,pos=.5] {CALIFICACI\'ON} (15,2.5)--(15,7)--(18,7)--(18,3.5) (8.4,0)--(8.4,1.5) (15,0)--(15,1.5) (10,1.5)--(10,2.5);
      \node[right] at (0,3.2) {\bf Alumno:}; \node[right] at (15,3.2) {\bf Carn\'e:};
      \node[right] at (0,2.2) {\bf Asignatura:};
      \node at (6,1.95) {\asignatura};
      \node[right] at (10,2.2) {\bf Profesor:};
      \node at (15,1.95) {\profesor};
      \node[right] at (0,1.2) {\bf Examen:};
      \draw (3.8,.9) rectangle (4.4,1.3); \node[left] at (3.8,1.1) {Parcial:};
      \draw (3.8,.2) rectangle (4.4,.6); \node[left] at (3.8,.4) {Previa:};
      \draw (7.4,.9) rectangle (8,1.3); \node[left] at (7.4,1.1) {Final:};
      \draw (7.4,.2) rectangle (8,.6); \node[left] at (7.4,.4) {Habilitaci\'on:};
       \node at (4, .4) {X}; % Quiz
      \node[right] at (10,.5) {8}; % Número de grupo
      \node[right] at (10,1.) {10 de febrero de 2017}; % Fecha de presentación
      \node[right] at (8.4,1.05) {\bf Fecha:}; \node[right] at (8.4,.45) {\bf Grupo:};
      \node[align=center,text width=3cm,font=\footnotesize] at (16.5,.75) {\centering\bf Use solo tinta\\y escriba claro};
    \end{tikzpicture}
  \end{minipage}
  Se permite usar calculadora de cualquier tipo mas no el uso de portátil o celulares en el examen, y puede disponer de sus apuntes de clase.
  Todo valor reportado debe aproximarse a 5 cifras significativas con redondeo simétrico (no es necesario en enteros y valores dados en enunciado) salvo que se indique lo contrario en el enunciado, con separador decimal consistente y unidades si lo requiere.
  En caso de reclamación solo cuenta lo que este en lapicero. Si algún elemento solicitado en teoría no puede realizarse, justifique por que no se puede realizar lo solicitado como respuesta. Toda respuesta debe estar adecuadamente justificada y/o con procedimiento, y resuelva las preguntas en orden.
\vspace{-.5cm}
  \begin{enumerate}[leftmargin=*,widest=9]
    \item Dado el siguiente pseudocódigo
\textbf{Inicio}
\(a\gets 5\)\\
\textbf{Mientras} \(a \geq 5\)\\
\hspace*{1cm}\(a \gets a + \ln(a)\)\\
\hspace*{1cm}\(b \gets \ln(a) \)\\
\textbf{Fin Mientras}\\
\textbf{Fin}\\
    \begin{enumerate}[label=\alph*]
    \item (\(0.7\)) ¿Cual es el valor de la variable \(b\) la segunda vez que es asignada?

\textbf{R/} \(b=2.1398\). La condición de ingreso se cumple al iniciar el pseudocódigo con el valor de \(a = 5\), por lo cual el valor de la variable pasa a ser \(a = a + \ln(a) = 6.6094\). A continuación se asigna en \(b\) el resultado de operarlo, pero este valor no influye en el próximo. Al volver a revisar la condición del ciclo, efectivamente se cumple, por lo cual el valor de \(a\) vuelve a cambiar conforme a \(a= a+\ln(a) =  8.4979\). En la siguiente linea \(b = \ln(a) = 2.1398 \).
    \item (\(0.3\)) ¿Cuantas veces se ejecutará el ciclo indicado en el pseudocódigo? Justifique la respuesta.

\textbf{R/} Infinitas (\(\infty\)) veces. Este ciclo no tendrá terminación pues se observa que la función asociada a la iteración del valor de \(a\) es una función monotonamente creciente, el valor de \(a\) siempre será mayor que \(5\) y por ende la condición del ciclo se cumplirá indefinidamente.
    \end{enumerate}
    \item La sumatoria del flujo de caja durante un periodo de tiempo medido en años para una inversión dada, puede ser modelada por \(f(x) = (x-5.5)^3 + x - 20\). Se denomina punto de equilibrio al tiempo para el cual la inversión realizada se recupera, siendo esto equivalente a que la sumatoria del flujo de caja sea cero. Use unidades si corresponde al reportar la respuesta final.
    \begin{enumerate}[label=\alph*]
    \item (\(0.7\)) Partiendo del intervalo [0, 10], determine a 3 iteraciones la aproximación del punto de equilibrio.

\textbf{R/} \(f(x)\) es un polinomio, por lo cual es continuo en todos los reales, y ademas, \(f(0)f(10)= -15120 < 0\). Así es posible usar el método cerrado.
\[
\begin{array}{|c|c|c|c|c|}
  \hline
  n & a & b & c & f(a)f(c)\\
  \hline
  1 & 0 & 10 & 5 & + \\
  \hline
  2 & 5 & 10 & 7.5000 & + \\
  \hline
  3 & 7.5000 & 10 & 8.7500 & - \\
  \hline
 \end{array}
\]
El punto de equilibrio se alcanza en aproximadamente \(8.7500\) años.
    \item (\(0.5\)) Sabiendo que una buena aproximación es \(7.8020\) años, determine el error absoluto de su aproximación.

\textbf{R/} \(E_a = |7.8020 - 8.7500| \text{ años} = 0.94800 \text{ años}\)
    \item (\(0.7\)) Con la información del literal anterior, determine el número de cifras significativas de su aproximación de manera formal.

\textbf{R/} Se requiere el error relativo para obtener las cifras significativas, para lo cual el literal anterior aporta los datos.
\[E_r = \left| \frac{E_a}{p}\right| = \frac{0.94800}{7.8020} = 0.12151\]
Ahora
\begin{eqnarray*}
0.12151 \cdot 10^0 \leq 0.5 \cdot 10^{1-n} \\ 0 = 1-n \\ n = 1
\end{eqnarray*}
Así, el numero de cifras significativas es 1.
\item (\(0.5\)) Reporte su resultado aproximado acorde a las cifras significativas que determino en el literal anterior.

\textbf{R/} = \(9 \cdot 10^0\) años.
\end{enumerate}
   \item Sea \(f(x) = \left\vert x \right\vert\), se desea conocer una aproximación de primer orden de la función para \(x=-0.1\) usando como punto central \(x_0=0.1\).
   \begin{enumerate}[label=\alph*]
    \item (\(0.8\)) ¿Cumple el teorema de aproximación de Taylor el problema planteado?

\textbf{R/} No cumple el teorema. Se observa que la primera derivada de la función en el intervalo requerido es discontinua por lo cual no se cumple una condición necesaria del teorema.
\item (\(0.8\)) Realice la aproximación indicada.

\textbf{R/} No es posible realizar la aproximación. Las aproximaciones de funciones por medio de expansiones de series de Taylor solo son posibles si cumplen las condiciones expuestas por el Teorema de Taylor, caso que no sucede aquí.
  \end{enumerate}
\end{enumerate}
\end{document}
