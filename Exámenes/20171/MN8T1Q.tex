\documentclass[12pt]{article}
\usepackage[letterpaper,margin={1.5cm}]{geometry}
\usepackage{amsmath, amssymb, amsfonts}
\usepackage[utf8]{inputenc}
\usepackage[T1]{fontenc}
\usepackage[spanish]{babel}
\usepackage{tikz}
\usepackage{graphicx,enumitem}
\usepackage{multicol}
\usepackage{hyperref}
\setlength{\marginparsep}{12pt} \setlength{\marginparwidth}{0pt} \setlength{\headsep}{.8cm} \setlength{\headheight}{15pt} \setlength{\labelwidth}{0mm} \setlength{\parindent}{0mm} \renewcommand{\baselinestretch}{1.15} \setlength{\fboxsep}{5pt} \setlength{\parskip}{3mm} \setlength{\arraycolsep}{2pt}
\renewcommand{\sin}{\operatorname{sen}}
\newcommand{\N}{\ensuremath{\mathbb{N}}}
\newcommand{\Z}{\ensuremath{\mathbb{Z}}}
\newcommand{\Q}{\ensuremath{\mathbb{Q}}}
\newcommand{\R}{\ensuremath{\mathbb{R}}}
\newcommand{\C}{\ensuremath{\mathbb{C}}}
\newcommand{\I}{\ensuremath{\mathbb{I}}}
\graphicspath{{../imagenes/}}
\allowdisplaybreaks{}

\raggedbottom{}
\setlength{\topskip}{0pt plus 2pt}
\newcommand{\profesor}{Edward Y. Villegas}
\newcommand{\asignatura}{M\'ETODOS NUM\'ERICOS}
\newcommand{\diff}[3]{\frac{d^{#3} #1}{d#2^{#3}}}
\newcommand{\pdiff}[3]{\frac{\partial^{#3} #1}{\partial#2^{#3}}}
\newcommand{\abs}[1]{\left| #1 \right|}
\begin{document}
  \pagestyle{empty}
  \begin{minipage}{\linewidth}
    \centering
    \begin{tikzpicture}[very thick,font=\small]
      \node at (2,6) {\includegraphics[width=3.5cm]{logoudem}};
      \node at (9.5,6) {\includegraphics[width=9cm]{cbudem}};
      \node[fill=white,draw=white,inner sep=1mm] at (9.5,5.05) {\bf Permanencia con calidad, Acompa\~nar para exigir};
      \node[fill=white,draw=white,inner sep=1mm] at (7.5,4.2) {\Large\bf DEPARTAMENTO DE CIENCIAS B\'ASICAS};
      \draw (0,0) rectangle (18,3.5);
      \draw (0,2.5)--(18,2.5) (0,1.5)--(18,1.5) (15,4.2)--(18,4.2) node[below,pos=.5] {CALIFICACI\'ON} (15,2.5)--(15,7)--(18,7)--(18,3.5) (8.4,0)--(8.4,1.5) (15,0)--(15,1.5) (10,1.5)--(10,2.5);
      \node[right] at (0,3.2) {\bf Alumno:}; \node[right] at (15,3.2) {\bf Carn\'e:};
      \node[right] at (0,2.2) {\bf Asignatura:};
      \node at (6,1.95) {\asignatura};
      \node[right] at (10,2.2) {\bf Profesor:};
      \node at (15,1.95) {\profesor};
      \node[right] at (0,1.2) {\bf Examen:};
      \draw (3.8,.9) rectangle (4.4,1.3); \node[left] at (3.8,1.1) {Parcial:};
      \draw (3.8,.2) rectangle (4.4,.6); \node[left] at (3.8,.4) {Previa:};
      \draw (7.4,.9) rectangle (8,1.3); \node[left] at (7.4,1.1) {Final:};
      \draw (7.4,.2) rectangle (8,.6); \node[left] at (7.4,.4) {Habilitaci\'on:};
       \node at (4, .4) {X}; % Quiz
      \node[right] at (10,.5) {8}; % Número de grupo
      \node[right] at (10,1.) {12 de marzo de 2017}; % Fecha de presentación
      \node[right] at (8.4,1.05) {\bf Fecha:}; \node[right] at (8.4,.45) {\bf Grupo:};
      \node[align=center,text width=3cm,font=\footnotesize] at (16.5,.75) {\centering\bf Use solo tinta\\y escriba claro};
    \end{tikzpicture}
  \end{minipage}
  \section{Condiciones}
  Para el desarrollo del primer taller-quiz (5\%) se recomienda el uso de una plataforma de programación cualquiera de la cual tenga acceso. Tenga en cuenta que algunas plataformas como excel no manejan por defecto el tipo de dato punto flotante sino decimal, por lo cual los resultados que se obtienen con celdas de excel pueden diferir significativamente de los resultados que se esperan como respuesta. Esta actividad presenta imposibilidad práctica para su realización a mano debido a la cantidad de iteraciones requeridas en algunos puntos.
La entrega de este taller se realiza a tráves de UVirtual en el espacio asignado para el taller 2 a más tardar, a las 23:59 del domingo 12 de marzo (domingo de la semana 7, anterior al inicio de parciales). Este taller se presenta individualmente en estilo quiz pero puede contar el desarrollo en grupo de los códigos o ejercicios de muestra y/o asesorías.
  Tendrá múltiples intentos diferidos por 6 horas y la duración máxima de un intento es 1:15 horas (15 minutos por pregunta).
  Todos los cálculos deben realizarse preservando las cifras significativas de la máquina (no redondear al interior del código) pero al reportarse en la respuesta debe realizarse acorde a las indicaciones del enunciado especifico y aplicando redondeo simétrico si lo requiere. Sus respuestas deben usar el separador decimal \verb-,-. En caso de no poder realizarse un problema por no cumplir las condiciones necesarias responda con la palabra \verb-No- o si el valor de una respuesta es infinito responda con \verb-oo- o \verb-inf-. Si el valor de una respuesta es un entero indique solo la parte entera.
  \section{Ejercicios}
  La siguiente es una lista de ejercicios de muestra que están presentes en el taller. Puede usar estos ejercicios para preparar sus códigos y adelantar algunas respuestas para optimizar el tiempo de presentación del taller.
\begin{itemize}
\item Dada la relación recursiva de \textbf{punto fijo} \(x_{n+1}=\sqrt{2x_n+3}\), encuentre el valor del punto fijo exacto.
\item Determine una aproximación de la raíz de \(\exp (-x) - 1\) en el intervalo dado por \(x \in \left[0{,}1, 0{,}2\right]\) con el \textbf{método de bisección} a 739 iteraciones y una tolerancia en la variable \(x\) de \(10^{-2}\).
\item Determine una aproximación de la raíz de \(\ln(x^2-2)\) en el intervalo dado por \(x \in \left[ 1{,}07, 2{,}07 \right]\) con el \textbf{método de bisección} con tolerancia en la evaluación de la función de \(10^{-12}\).
\item Determine una aproximación de la raíz de \(\tan(x - \pi)\) en el intervalo dado por \(x \in \left[ 1{,}4, 1{,}8 \right]\) con el \textbf{método de bisección} a 421 iteraciones y tolerancia en la variable \(x\) de \(10^{-6}\).
\item Dada la relación recursiva de \textbf{punto fijo} \(x_{n+1}=\sqrt{2x_n+3}\), encuentre el mayor valor que cumple con el teorema de existencia y unicidad.
\item Dada la relación recursiva de \textbf{punto fijo} \(x_{n+1}=\sqrt{2x_n+3}\), encuentre el menor valor que cumple con el teorema de existencia y unicidad.
\item Dada la aproximación \(2{,}71826171875\) de \(\exp(1)\) (\textit{ número de Euler}), determine cuantas cifras significativas posee dicha aproximación.
\item Dada la aproximación \(1{,}615789436\) de \(\phi\) (\textit{proporción áurea}), determine cuantas cifras significativas posee dicha aproximación.
\item Dada la aproximación \(3{,}144549672\) de \(\pi\), determine cuantas cifras significativas posee dicha aproximación.
\item Dada la aproximación \(1{,}41976642\) de \(\sqrt{2}\), determine cuantas cifras significativas posee dicha aproximación.
\item Dada la aproximación \(1{,}731005146\) de \(\sqrt{3}\), determine cuantas cifras significativas posee dicha aproximación.
\item Aproxime la raíz de \(f(x) = (x-5.5)^3 + x - 20\)  usando el \textbf{método de bisección} en el intervalo \(x \in \left[0, 10\right]\) con 1500 iteraciones. Use 8 cifras significativas al reportar.
\item Aproxime la raíz de \(\exp(x^2 - 2) - 1\) usando el \textbf{método de bisección} con \(n=941\) en el intervalo \(x \in \left[1{,}4, 1{,}5\right]\) y una tolerancia absoluta tanto en la variable \(x\) como en la evaluación de la función de \(10^{-8}\). Use 12 cifras significativas al responder.
\item Aproxime la raíz de \(\ln(x^2 - 2)\) usando el \textbf{método de bisección} en el intervalo \(x \in \left[1{,}7, 1{,}8\right]\) y una tolerancia absoluta en la variable \(x\) de \(10^{-6}\). Use 9 cifras significativas al responder.
\item Aproxime la raíz de \(\ln(\ln(x))\)  usando el \textbf{método de bisección} en el intervalo \(x \in \left[2{,}6, 2{,}8\right]\) con 561 iteraciones y una tolerancia absoluta en la variable \(x\) de \(10^{-4}\) y en la evaluación de la función de \(10^{-7}\). Use 8 cifras significativas al reportar.
\item Aproxime la raíz de \(\tan(x - \pi)\) usando el \textbf{método de bisección} en el intervalo \(x \in \left[3{,}1, 3{,}2\right]\) y una tolerancia absoluta en la evaluación de la función de \(10^{-5}\). Use 8 cifras significativas al responder.
\item Aproxime la raíz de \(\tan(\tan(x))\)  usando el \textbf{método de bisección} en el intervalo \(x \in \left[3{,}0, 3{,}2\right]\) con 86 iteraciones y una tolerancia absoluta en la variable \(x\) de \(10^{-9}\). Use 12 cifras significativas al reportar.
\item Aproxime la raíz de \(\ln(x^2 - 2)\)  usando el \textbf{método de Newton} con valor inicial de \(1{,}79\) y una tolerancia absoluta en la evaluación de la función de \(10^{-6}\). Use 12 cifras significativas al responder.
\item Aproxime la raíz de \(\tan(\tan(x))\)  usando el \textbf{método de Newton} con el valor inicial de \(3{,}023\) con 14 iteraciones y una tolerancia absoluta en la variable \(x\) de \(10^{-9}\). Use 16 cifras significativas al reportar.
\item Dada la relación recursiva de \textbf{punto fijo} \(x_{n+1}=\sqrt{2x_n+3}\), encuentre el valor del punto fijo iniciando con \(x_0=4\) y usando 18 iteraciones y una tolerancia de \(10^{-6}\). Reporte con 10 cifras significativas.
\item Dada la relación recursiva de \textbf{punto fijo} \(x_{n+1}=\sqrt{2x_n+3}\), encuentre el valor del punto fijo iniciando con \(x_0=4\) y usando 18 iteraciones y una tolerancia de \(10^{-2}\). Reporte con 10 cifras significativas.
\item Dada la relación recursiva de \textbf{punto fijo} \(x_{n+1} = \frac{x_n + 1/x_n}{2}\), encuentre el valor del punto fijo con precisión de \(10^{-10}\) usando como valor inicial \(1{,}5\). Reporte el resultado con 10 cifras significativas.
\item Dada la relación recursiva de \textbf{punto fijo} \(x_{n+1} = \frac{x_n + 1/x_n}{2}\), encuentre el valor del punto fijo con precisión de \(10^{-2}\) usando como valor inicial \(1{,}5\). Reporte el resultado con 10 cifras significativas.
\item Dado el polinomio \(x^{100} - 5x^{50} -1\), encuentre la aproximación a una de las raíces reales usando como valor inicial \(0{,}95\) y una tolerancia absoluta tanto en la variable \(x\) como en la evaluación de la función de \(10^{-5}\). Use el \textbf{método de Newton Horner} y reporte la respuesta con 12 cifras significativas.
\item Dado el polinomio \(x^{100} - 5x^{50} -1\), encuentre la aproximación a una de las raíces reales usando como valor inicial \(0{,}95\) y una tolerancia absoluta tanto en la variable \(x\) como en la evaluación de la función de \(10^{-5}\). Use el \textbf{método de Newton Horner} y reporte la respuesta con 10 cifras significativas.
\item Dado el polinomio \(\sum\limits_{n=0}^{10}(n-5)(-1)^n x^{2n}\), encontrar el coeficiente de grado 14 del polinomio cociente que se usa para buscar la segunda raíz en orden de extracción  usando el \textbf{método de Newton Horner} con valor inicial de \(x=0{,}7\) y 30 iteraciones. Reporte la respuesta con 14 cifras significativas.
\item Dado el polinomio \(\sum\limits_{n=0}^{10}(n-5)(-1)^n x^{2n}\), encontrar la primera raíz real en orden de extracción usando el \textbf{método de Newton Horner} con valor inicial de \(x=0{,}7\) y 30 iteraciones. Reporte la respuesta con 14 cifras significativas.
\item Dado el polinomio \(\sum\limits_{n=0}^{10}(n-5)(-1)^n x^{2n}\), encontrar la segunda raíz real en orden de extracción usando el \textbf{método de Newton Horner} con valor inicial de \(x=0{,}7\) y 30 iteraciones. Reporte la respuesta con 14 cifras significativas.
\item Dado el polinomio \(P(x) = x^3 - 5x^2 + 17x -13\) encuentre el coeficiente cuadrático del polinomio residual al evaluar \(1{,}003\) con el \textbf{algoritmo de Horner}. Reporte con 8 cifras significativas.
\item Dado el polinomio \(P(x) = x^3 - 5x^2 + 17x -13\) encuentre una raíz del polinomio usando como valor inicial \(1{,}5\) con 2462 iteraciones y el \textbf{método de Newton Horner}. Reporte con 10 cifras significativas.
\item Dado el polinomio \(P(x) = x^3 - 5x^2 + 17x -13\) encuentre una raíz del polinomio usando como valor inicial \(1{,}5\) con 2462 iteraciones y una tolerancia en la variable \(x\) de \(10^{-3}\). Use \textbf{método de Newton Horner}. Reporte con 12 cifras significativas.
\item Dado el polinomio \(p_6(x)=x^6+4x^5-72x^4-214x^3+1127x^2+1602x-5040\), encuentre el coeficiente grado 4 del polinomio residual al evaluar en \(-8{,}000000001990397\) con el \textbf{algoritmo de Horner}. Reporte a 12 cifras significativas la respuesta.
\item Dado el polinomio \(p_6(x)=x^6+4x^5-72x^4-214x^3+1127x^2+1602x-5040\), encuentre la cuarta raíz en orden de obtención usando el \textbf{método de Newton Horner} y como valor inicial \(-15\) con tolerancia absoluta en la variable \(x\) de \(10^{-2}\). Reporte a 8 cifras significativas la respuesta.
\item Dado el polinomio \(p_6(x)=x^6+4x^5-72x^4-214x^3+1127x^2+1602x-5040\), encuentre una raíz usando el \textbf{método de Newton Horner} y como valor inicial \(-15\) con tolerancia absoluta en la evaluación del polinomio de \(10^{-8}\). Reporte a 8 cifras significativas la respuesta.
\item Dado el polinomio \(p_6(x)=x^6+4x^5-72x^4-214x^3+1127x^2+1602x-5040\), encuentre una raíz usando el \textbf{método de Newton Horner} y como valor inicial \(-15\) con tolerancia absoluta en la evaluación del polinomio de \(2\). Reporte a 15 cifras significativas la respuesta.
\end{itemize}
\end{document}
