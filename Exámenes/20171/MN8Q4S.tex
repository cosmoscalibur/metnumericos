\documentclass[12pt]{article}

\usepackage[letterpaper,margin={1.5cm}]{geometry}
\usepackage{amsmath, amssymb, amsfonts}

\usepackage[utf8]{inputenc}
\usepackage[T1]{fontenc}
\usepackage[spanish]{babel}
\usepackage{tikz}
\usepackage{graphicx,enumitem}
\usepackage{multicol}
\usepackage{hyperref}

\setlength{\marginparsep}{12pt} \setlength{\marginparwidth}{0pt} \setlength{\headsep}{.8cm} \setlength{\headheight}{15pt} \setlength{\labelwidth}{0mm} \setlength{\parindent}{0mm} \renewcommand{\baselinestretch}{1.15} \setlength{\fboxsep}{5pt} \setlength{\parskip}{3mm} \setlength{\arraycolsep}{2pt}

\renewcommand{\sin}{\operatorname{sen}}
\newcommand{\N}{\ensuremath{\mathbb{N}}}
\newcommand{\Z}{\ensuremath{\mathbb{Z}}}
\newcommand{\Q}{\ensuremath{\mathbb{Q}}}
\newcommand{\R}{\ensuremath{\mathbb{R}}}
\newcommand{\C}{\ensuremath{\mathbb{C}}}
\newcommand{\I}{\ensuremath{\mathbb{I}}}

\graphicspath{{../imagenes/}{imagenes/}{./}}
\allowdisplaybreaks{}
\raggedbottom{}
\setlength{\topskip}{0pt plus 2pt}

\newcommand{\profesor}{Edward Y. Villegas}
\newcommand{\asignatura}{MÉTODOS NUMÉRICOS}


\newcommand{\diff}[3]{\frac{d^{#3} #1}{d#2^{#3}}}
\newcommand{\pdiff}[3]{\frac{\partial^{#3} #1}{\partial#2^{#3}}}
\newcommand{\abs}[1]{\left| #1 \right|}

\begin{document}
  \pagestyle{empty}
  \begin{minipage}{\linewidth}
    \centering
    \begin{tikzpicture}[very thick,font=\small]
      \node at (2,6) {\includegraphics[width=3.5cm]{logoudem}};
      \node at (9.5,6) {\includegraphics[width=9cm]{cbudem}};
      \node[fill=white,draw=white,inner sep=1mm] at (9.5,5.05) {\bf Permanencia con calidad, Acompañar para exigir};
      \node[fill=white,draw=white,inner sep=1mm] at (7.5,4.2) {\Large\bf DEPARTAMENTO DE CIENCIAS BÁSICAS};
      \draw (0,0) rectangle (18,3.5);
      \draw (0,2.5)--(18,2.5) (0,1.5)--(18,1.5) (15,4.2)--(18,4.2) node[below,pos=.5] {CALIFICACIÓN} (15,2.5)--(15,7)--(18,7)--(18,3.5) (8.4,0)--(8.4,1.5) (15,0)--(15,1.5) (10,1.5)--(10,2.5);
      \node[right] at (0,3.2) {\bf Alumno:}; \node[right] at (15,3.2) {\bf Carné:};
      \node[right] at (0,2.2) {\bf Asignatura:};
      \node at (6,1.95) {\asignatura};
      \node[right] at (10,2.2) {\bf Profesor:};
      \node at (15,1.95) {\profesor};
      \node[right] at (0,1.2) {\bf Examen:};
      \draw (3.8,.9) rectangle (4.4,1.3); \node[left] at (3.8,1.1) {Parcial:};
      \draw (3.8,.2) rectangle (4.4,.6); \node[left] at (3.8,.4) {Previa:};
      \draw (7.4,.9) rectangle (8,1.3); \node[left] at (7.4,1.1) {Final:};
      \draw (7.4,.2) rectangle (8,.6); \node[left] at (7.4,.4) {Habilitación:};
       \node at (4, .4) {X}; % Quiz
      \node[right] at (10,.5) {8}; % Número de grupo
      \node[right] at (10,1.) {12 de mayo de 2017}; % Fecha de presentación
      \node[right] at (8.4,1.05) {\bf Fecha:}; \node[right] at (8.4,.45) {\bf Grupo:};
      \node[align=center,text width=3cm,font=\footnotesize] at (16.5,.75) {\centering\bf Use solo tinta\\y escriba claro};
    \end{tikzpicture}
  \end{minipage}

  Se permite usar calculadora de cualquier tipo mas no el uso de portátil o celulares en el examen, y puede disponer de sus apuntes de clase.
  Todo valor reportado debe aproximarse a 5 cifras significativas con redondeo simétrico (no es necesario en enteros y valores dados en enunciado) salvo que se indique lo contrario en el enunciado, con separador decimal consistente y unidades si lo requiere.
  En caso de reclamación solo cuenta lo que este en lapicero. Si algún elemento solicitado en teoría no puede realizarse, justifique por que no se puede realizar lo solicitado como respuesta. Toda respuesta debe estar adecuadamente justificada y/o con procedimiento, y resuelva las preguntas en orden.

\vspace{-.5cm}
  \begin{enumerate}[leftmargin=*,widest=9]

    \item Dado el P.V.I. siguiente:

    \[
      \diff{p(s)}{s}{} = s\sqrt{p(s)};\qquad
      p(2) = 1.
    \]

    \begin{enumerate}[label=\alph*]
    \item (\(1.5\)) Demuestre que el P.V.I. dado cumple con poseer solución única asegurada sin importar cual sea el valor final del intervalo de \(s\).
    \item (\(0.8\)) Usando el método de Runge Kutta de orden 4, aproxime la solución de \(s=2.2\) usando un solo paso en \(s\). Parta de que cumple la solución única asegurada acorde al literal anterior.
    \item (\(0.7\)) Sabiendo que la solución exacta del P.V.I. es \(p(s)=s^4/16\), determine el número de cifras significativas que posee su aproximación y reescribala acorde a eso.
    \end{enumerate}

    \item En una región habitan una población de zorros (\(z(t)\)) y una población de conejos (\(c(t)\)). Los leones en esta región solo se alimentan de conejos y los conejos no poseen dificultad para conseguir su alimento.
    Sus poblaciones son modeladas por las ecuaciones de Lotka Volterra dadas a continuación:
    \[
\diff{c(t)}{t}{} = \frac{0.2}{\text{año}} c(t) - \frac{0.7}{\text{año}} c(t) z(t);\qquad
\diff{z(t)}{t}{} = -\frac{0.4}{\text{año}} z(t) + \frac{0.5}{\text{año}} c(t) z(t)
    \]
    La población inicial de conejos es de 57 individuos y la población inicial de zorros es de 23 individuos.
    \begin{enumerate}[label=\alph*]
    \item (\(0.5\)) Lleve el P.V.I. de sistema de ecuaciones diferenciales dado a la forma de un P.V.I. de una ecuación diferencial vectorial.
    \item (\(0.5\)) Valide que el P.V.I. dado posee solución única asegurada.
    \item (\(1.0\)) Encuentre con un solo paso temporal la aproximación de la población de ambas especies transcurrido medio año. Use el método numérico de su preferencia.
    \end{enumerate}

\end{enumerate}

\end{document}
