\documentclass[12pt]{article}
\usepackage[letterpaper,margin={1.5cm}]{geometry}
\usepackage{amsmath, amssymb, amsfonts}
\usepackage[utf8]{inputenc}
\usepackage[T1]{fontenc}
\usepackage[spanish]{babel}
\usepackage{tikz}
\usepackage{graphicx,enumitem}
\usepackage{multicol}
\setlength{\marginparsep}{12pt} \setlength{\marginparwidth}{0pt} \setlength{\headsep}{.8cm} \setlength{\headheight}{15pt} \setlength{\labelwidth}{0mm} \setlength{\parindent}{0mm} \renewcommand{\baselinestretch}{1.15} \setlength{\fboxsep}{5pt} \setlength{\parskip}{3mm} \setlength{\arraycolsep}{2pt}
\renewcommand{\sin}{\operatorname{sen}}
\newcommand{\N}{\ensuremath{\mathbb{N}}}
\newcommand{\Z}{\ensuremath{\mathbb{Z}}}
\newcommand{\Q}{\ensuremath{\mathbb{Q}}}
\newcommand{\R}{\ensuremath{\mathbb{R}}}
\newcommand{\C}{\ensuremath{\mathbb{C}}}
\newcommand{\I}{\ensuremath{\mathbb{I}}}
\graphicspath{{../imagenes/}{imagenes/}{..}}
\allowdisplaybreaks{}
\raggedbottom{}
\setlength{\topskip}{0pt plus 2pt}
\newcommand{\profesor}{Edward Y. Villegas}
\newcommand{\asignatura}{M\'ETODOS NUM\'ERICOS}
\begin{document}
  \pagestyle{empty}
  \begin{minipage}{\linewidth}
    \centering
    \begin{tikzpicture}[very thick,font=\small]
      \node at (2,6) {\includegraphics[width=3.5cm]{logoudem}};
      \node at (9.5,6) {\includegraphics[width=9cm]{cbudem}};
      \node[fill=white,draw=white,inner sep=1mm] at (9.5,5.05) {\bf Permanencia con calidad, Acompa\~nar para exigir};
      \node[fill=white,draw=white,inner sep=1mm] at (7.5,4.2) {\Large\bf DEPARTAMENTO DE CIENCIAS B\'ASICAS};
      \draw (0,0) rectangle (18,3.5);
      \draw (0,2.5)--(18,2.5) (0,1.5)--(18,1.5) (15,4.2)--(18,4.2) node[below,pos=.5] {CALIFICACI\'ON} (15,2.5)--(15,7)--(18,7)--(18,3.5) (8.4,0)--(8.4,1.5) (15,0)--(15,1.5) (10,1.5)--(10,2.5);
      \node[right] at (0,3.2) {\bf Alumno:}; \node[right] at (15,3.2) {\bf Carn\'e:};
      \node[right] at (0,2.2) {\bf Asignatura:};
      \node at (6,1.95) {\asignatura};
      \node[right] at (10,2.2) {\bf Profesor:};
      \node at (15,1.95) {\profesor};
      \node[right] at (0,1.2) {\bf Examen:};
      \draw (3.8,.9) rectangle (4.4,1.3); \node[left] at (3.8,1.1) {Parcial:};
      \draw (3.8,.2) rectangle (4.4,.6); \node[left] at (3.8,.4) {Previa:};
      \draw (7.4,.9) rectangle (8,1.3); \node[left] at (7.4,1.1) {Final:};
      \draw (7.4,.2) rectangle (8,.6); \node[left] at (7.4,.4) {Habilitaci\'on:};
      \node at (4,0.5) {X}; % Variar segun el tipo de examen
      \node[right] at (8.4,1.05) {\bf Fecha:}; \node[right] at (8.4,.45) {\bf Grupo:};
      \node[align=center,text width=3cm,font=\footnotesize] at (16.5,.75) {\centering\bf Use solo tinta\\y escriba claro};
    \end{tikzpicture}
  \end{minipage}
  \begin{enumerate}[leftmargin=*,widest=9]
    \item \textbf{Punto fijo} Para la relación recursiva de punto fijo \[x_{n+1} = \frac{x_n}{2} + \frac{1}{x_n},\]
    \begin{itemize}
    \item[$0.5$] ¿Cumple el criterio de punto fijo para todo $x_0 = 1$? Justifique.
    \vspace{3.5cm}
    \item[$1.0$] ¿A que valor converge la relación recursiva?.
    \vspace{2cm}
    \end{itemize}
 \item Dados los puntos $P=\lbrace (-1, 4), (1, -6), (4, -3) \rbrace$ extraídos de una función continua en el intervalo $\left[-1, 4\right]$.
    \begin{itemize}
    \item[$0.5$] \textbf{Interpolación} Determine el polinomio interpolante de grado mínimo para el conjunto de puntos $P$.
    \vspace{5cm}
    \item[$0.5$] \textbf{Diferenciación} Usando diferencia central y con $\Delta x = 0.2$, determine el resultado aproximado de la primera iteración de M\"uller con semilla en $x_0=0$.
    \[
    x_1 = -\frac{2f(x_0)}{f\prime(x_0)+\sqrt{f^\prime(x_0)^2 - 2f(x_0)f^{\prime\prime}(x_0)}}.
    \]
    \vspace{4cm}
    \item[$0.5$] \textbf{Raíces} Con el polinomio interpolante hallado, encuentre la aproximación de la raíz de la función al interior del intervalo con tolerancia de $0.1$ con el método de su preferencia.
    \vspace{5cm}
    \end{itemize}
    \item Dados los puntos de la forma $(x_i, f(x_i), f^\prime (x_i))$ de una función desconocida, $Q=\lbrace (0, 3, 1), (1, 4, 0) \rbrace$.
    \begin{itemize}
    \item[$1.0$] \textbf{Interpolación} Determine el polinomio interpolante de Hermite para la información dada.
    \vspace{5cm}
    \item[$1.0$] \textbf{Integración} Usando $\Delta x = 0.1$, estime la integral del polinomio interpolante hallado en el intervalo $\left[0.2, 0.5\right]$, $\int_{0.2}^{0.5} H_{3}(x)dx$.
    \end{itemize}
  \end{enumerate}
  \clearpage
  \textbf{Ecuaciones}
\begin{align*}
  \text{Cerrados: } & f(a)f(b)<0 &  \\
  \begin{array}{c}
  \text{Incremental:}\\
  h = \frac{b-a}{n} \\
  x_{i+1} = x_i + ih\\
  x_r = \frac{x_{i+1}-x_i}{2}
  \end{array}
  & \qquad \qquad &
  \begin{array}{c}
  \text{Bisección:}\\
  c = \frac{b-a}{2}\\
  b = c, f(a)f(c)<0\\
  a = c, f(b)f(c)<0
  \end{array}
\end{align*}
\begin{align*}
  \text{Abiertos - Punto fijo:}& \qquad \qquad |g^\prime (x)| < 1 & x_{i+1} = g(x_i) \\
  \text{Newton} & & \text{Secante} \\
  x_{i+1} = x_i - \frac{f(x_i)}{f^\prime(x)} & \qquad \qquad &
    x_{i+1} = x_i - \frac{f(x_i)\Delta x}{f(x_i+\Delta x)-f(x_i)}
\end{align*}
  \textit{Diferenciación:} $ \Delta x = \frac{b-a}{n}$\\
  \textit{Diferencias divididas:}
  \begin{eqnarray*}
  f([x_i, x_{i+1}]) = \frac{f(x_{i+1})-f(x_i)}{x_{i+1}-x_i}\\
  f([x_i, \ldots, x_k]) = \frac{f([x_{i+1}, \ldots, x_k]) - f([x_i, \ldots, x_{k-1}])}{x_k - x_i}
  \end{eqnarray*}
  \textit{Diferencia central:} $$\frac{d}{dx}f(x_i) \approx \frac{f([x_{i-1}, x_i])+f([x_{i}, x_{i+1}])}{2} $$
  \textit{Interpolación:} $n+1 \rightarrow P_n(x)$\\
  \textit{Lagrange:} $$L_n(x) = \sum_{i=0}^n \left(y_i\prod_{\substack{j=0 \\ i \neq j}}^n \frac{x-x_j}{x_i - x_j}\right)$$
  \textit{Newton:} $$N_n(x) = \sum_{i=0}^n \left(f([x_0, \ldots, x_i]) \prod_{j=0}^{i-1}(x-x_j)\right)$$
  \textit{Hermite:}
  \begin{eqnarray*}
  n+1 \rightarrow H_{2n+1}(x)\\
  z_{2i} = x_i\\
  z_{2i+1} = x_i\\
  H_{2n+1}(x) = \sum_{j=0}^{2n+1}f([z_0, \ldots, z_j])\prod_{k=0}^{j-1}(x - z_k)
  \end{eqnarray*}
  \textit{Integración:}
  \begin{eqnarray*}
   \Delta x = \frac{b-a}{n}\\
   x_i = a + i \Delta x
  \end{eqnarray*}
  \textit{Trapecio:}
  $$\int_a^b f(x)dx \approx \sum_{i=0}^{n-1} \frac{f(x_i) + f(x_{i+1})}{2}\Delta x$$
  \textit{Simpson 1/3:}
  $$ \int_a^b f(x)dx \approx \frac{b-a}{6} \left[f(a) + 4f\left(\frac{a+b}{2} \right) + f(b) \right] $$\\
  \textit{Simpson 3/8:}
  $$ \int_a^b f(x)dx \approx \frac{b-a}{8} \left[f(a) + 3f\left(\frac{2a+b}{3}\right) + 3f\left(\frac{a+2b}{3}\right) + f(b) \right] $$
\end{document}
