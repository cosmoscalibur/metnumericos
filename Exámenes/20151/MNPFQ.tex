\documentclass[12pt]{article}

\usepackage[letterpaper,margin={1.5cm}]{geometry}
\usepackage{amsmath, amssymb, amsfonts}

\usepackage[utf8]{inputenc}
\usepackage[T1]{fontenc}
\usepackage[spanish]{babel}
\usepackage{tikz}
\usepackage{graphicx,enumitem}
\usepackage{multicol}

\setlength{\marginparsep}{12pt} \setlength{\marginparwidth}{0pt} \setlength{\headsep}{.8cm} \setlength{\headheight}{15pt} \setlength{\labelwidth}{0mm} \setlength{\parindent}{0mm} \renewcommand{\baselinestretch}{1.15} \setlength{\fboxsep}{5pt} \setlength{\parskip}{3mm} \setlength{\arraycolsep}{2pt}

\renewcommand{\sin}{\operatorname{sen}}
\newcommand{\N}{\ensuremath{\mathbb{N}}}
\newcommand{\Z}{\ensuremath{\mathbb{Z}}}
\newcommand{\Q}{\ensuremath{\mathbb{Q}}}
\newcommand{\R}{\ensuremath{\mathbb{R}}}
\newcommand{\C}{\ensuremath{\mathbb{C}}}
\newcommand{\I}{\ensuremath{\mathbb{I}}}

\graphicspath{{../imagenes/}}

\allowdisplaybreaks

\raggedbottom
\setlength{\topskip}{0pt plus 2pt}

%\spanishdecimal{.}

\newcommand{\profesor}{Edward Y. Villegas}
\newcommand{\asignatura}{M\'ETODOS NUM\'ERICOS}

% \DeclareMathOperator{\sen}{sen}
% \renewcommand{\sin}{\sen}

\begin{document}
  \pagestyle{empty}
  \begin{minipage}{\linewidth}
    \centering
    \begin{tikzpicture}[very thick,font=\small]
%       \draw[help lines,step=5mm,red] (0,0) grid (18,7);
      \node at (2,6) {\includegraphics[width=3.5cm]{logoudem}};
      \node at (9.5,6) {\includegraphics[width=9cm]{cbudem}};
      \node[fill=white,draw=white,inner sep=1mm] at (9.5,5.05) {\bf Permanencia con calidad, Acompa\~nar para exigir};
      \node[fill=white,draw=white,inner sep=1mm] at (7.5,4.2) {\Large\bf DEPARTAMENTO DE CIENCIAS B\'ASICAS};
      \draw (0,0) rectangle (18,3.5);
      \draw (0,2.5)--(18,2.5) (0,1.5)--(18,1.5) (15,4.2)--(18,4.2) node[below,pos=.5] {CALIFICACI\'ON} (15,2.5)--(15,7)--(18,7)--(18,3.5) (8.4,0)--(8.4,1.5) (15,0)--(15,1.5) (10,1.5)--(10,2.5);
      \node[right] at (0,3.2) {\bf Alumno:}; \node[right] at (15,3.2) {\bf Carn\'e:};
      \node[right] at (0,2.2) {\bf Asignatura:};
      \node at (6,1.95) {\asignatura};
      \node[right] at (10,2.2) {\bf Profesor:};
      \node at (15,1.95) {\profesor};
      \node[right] at (0,1.2) {\bf Examen:};
      \draw (3.8,.9) rectangle (4.4,1.3); \node[left] at (3.8,1.1) {Parcial:};
      \draw (3.8,.2) rectangle (4.4,.6); \node[left] at (3.8,.4) {Previa:};
      \draw (7.4,.9) rectangle (8,1.3); \node[left] at (7.4,1.1) {Final:};
      \draw (7.4,.2) rectangle (8,.6); \node[left] at (7.4,.4) {Habilitaci\'on:};
      % \node at (4,1.1) {X}; % Parcial
      % \node at (4, .4) {X}; % Quiz
      \node at (7.6, 1.1) {X}; % Final
      \node[right] at (10,.5) {}; % Número de grupo
      \node[right] at (10,1.) {3 de junio de 2015}; % Fecha de presentación
      \node[right] at (8.4,1.05) {\bf Fecha:}; \node[right] at (8.4,.45) {\bf Grupo:};
      \node[align=center,text width=3cm,font=\footnotesize] at (16.5,.75) {\centering\bf Use solo tinta\\y escriba claro};
    \end{tikzpicture}
  \end{minipage}
  
  Para la realización de este examen, puede usar calculadora científica o portátil.
  Al final, encontrará una hoja de formulas para su ayuda o use sus notas digitales (en caso de usar portátil).
  Si necesita espacio adicional, use el respaldo de las hojas para el procedimiento,
  e \textbf{indique claramente la respuesta final de cada pregunta al lado del enunciado}. No se pueden usar hojas extras.
  Use como mínimo aproximaciones a \textbf{3 DECIMALES} en todas las operaciones.

  \begin{enumerate}[leftmargin=*,widest=9]
    %% Punto 1
    \item \textbf{Runge Kutta 4} Dada la ecuación diferencial:
    \[
    \frac{dy}{dt} = t + \frac{y}{t}; \quad y(0.1)=0.2
    \]
    \begin{enumerate}[label=\alph*]
    \item ($0.50$) Determine la constante de Lipschitz asociada a la función de la ecuación diferencial. ¿Posee solución única asegurada?
    \vspace{2cm}
    
    \item ($0.50$) Con tamaño de paso \(h=0.2\), use el método RK4 para aproximar la solución en \(y(0.3)\).
    \vspace{3cm}
    
    \item ($0.25$) Calcule el error relativo en \(y(0.3)\), si la solución analítica de la ecuación diferencial es \(y(t) = t(t+1.9)\).
    \vspace{2cm}
    
    \end{enumerate}

    %% Punto 2
    \item Dado el sistema matricial
    \[
    \begin{pmatrix}
    -4 & 5 \\ 1 & 2
    \end{pmatrix} \vec{x} = \begin{pmatrix}
    1 \\ 3
    \end{pmatrix}
    \]
    \begin{enumerate}[label=\alph*]
    \item ($0.50$) Determine la matriz \(T_G\) asociada al método de Gauss-Seidel.
    \vspace{2cm}
    
    \item ($0.25$) Calcule el radio espectral \(\rho(T_G)\).
    \vspace{2cm}
    
    \item ($0.50$) Para el método de SOR con \(\omega=0.7\), el radio espectral es \(\rho(T_{SOR}) = 0.3\). ¿Cual de los dos métodos converge más rápidamente?
    \vspace{1cm}
    
    \end{enumerate}
   
    %% Pregunta 3
    \item \textbf{Método de Jacobi} Dado el sistema de ecuaciones lineales
    \begin{align*}
    3x_1 - 0.1x_2 - 0.2x_3 & = 7.85 \\
    0.1x_1 + 7x_2 - 0.3x_3 & = -19.3 \\
    0.3x_1 - 0.2x_2 + 10x_3 & = 71.4
    \end{align*}
    
    \begin{enumerate}[label=\alph*]

    \item ($0.50$) Determine la matriz \(T_J\) y el vector \(\vec{c}_J\) para el método de Jacobi, para el sistema \(A\vec{x}=\vec{b}\) equivalente al sistema de ecuaciones.
    \vspace{2cm}
    
    \item ($0.50$) Usando como aproximación inicial el vector nulo, muestre la solución aproximada tras dos iteraciones.
    \vspace{5cm}
    
    \item ($0.50$) Calcule el error absoluto de la aproximación usando la norma 3 (\(l_3\)) con respecto a la solución exacta. La solución exacta es el vector 
    \( \begin{pmatrix} 3 \\ -2.5 \\ 7 \end{pmatrix} \).
    \vspace{4cm}
    
    \end{enumerate}
    
    %% Pregunta 4: Falso-Verdadero.
    \item ($1.00$) \textbf{Falso Verdadero} Relacione a continuación de cada afirmación su valor de falso (\textbf{F}) o verdadero (\textbf{V}) según corresponda. Justifique brevemente solo si es falsa.
    
    \begin{itemize}
    \item La ecuación diferencial \(y^{\prime}(t)= t^2 + 2t\) es soluble exactamente (salvo por error de redondeo) con el método de Taylor orden 3. ( )
    \vspace{1cm}
    \item Un sistema representado por una matriz estrictamente diagonal dominante converge a la solución única por el método de Gauss-Seidel. ( )
    \vspace{1cm}
    \item Es posible aplicar el método de Jacobi existiendo ceros en la diagonal de la matriz del sistema (\(a_{ii}=0\)), sin necesidad de aplicar estrategias de pivoteo. ( )
    \vspace{1cm}
    \item La convergencia del método SOR a la solución del sistema matricial es dependiente de la selección del valor del parámetro de relajación. ( )
%    \vspace{1cm}
    \end{itemize}
    
  \end{enumerate}
  
  \textbf{Formulas}
  
%  \[\epsilon_r = \left\vert \frac{p - p^*}{p} \right\vert \]
%  \textit{Ecuaciones diferenciales}
%  \[
%  \left\vert \frac{\partial f(t, y)}{\partial y} \right\vert \leq L
%  \]
%  
  \begin{eqnarray*}
  \epsilon_r & = & \left\vert \frac{p - p^*}{p} \right\vert \\
  \left\vert \frac{\partial f(t, y)}{\partial y} \right\vert & \leq & L \\
  y_{i+1} & = & y_i + \frac{K1 + 2K2 + 2K3 + K4}{6} \\
  K1 & = & hf(t_i, y_i) \\
  K2 & = & hf(t_i + h/2, y_i + K1/2) \\
  K3 & = & hf(t_i + h/2, y_i + K2/2) \\
  K4 & = & hf(t_{i+1}, y_i + K3)\\
%  \end{eqnarray*}
%  
%  \textit{Sistemas matriciales}
%  
%  \begin{eqnarray*}
  P_{T}(\lambda) & = & \det(A-\lambda I) = 0 \\
  \rho(T) & = & \max \lbrace \vert \lambda_i \vert \rbrace \\
  \Vert \vec{x} \Vert_{p} & = & \sqrt[p]{\sum_{i=0}^{n-1} \vert x_i \vert^p} \\
  \left\vert z \right\vert & = & \sqrt{zz^*} \\
  A & = & D - L - U \\
  T_J & = & D^{-1}(L+U) \\
  \vec{c}_J & = & D^{-1}\vec{b} \\
  T_G & = & (D-L)^{-1}U \\
  \vec{x}^{(k+1)} & = & T\vec{x}^{(k)} + \vec{c}
  \end{eqnarray*}
  
  
  

\end{document}
