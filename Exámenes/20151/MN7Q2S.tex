\documentclass[12pt]{article}
\usepackage[letterpaper,margin={1.5cm}]{geometry}
\usepackage{amsmath, amssymb, amsfonts}
\usepackage[utf8]{inputenc}
\usepackage[T1]{fontenc}
\usepackage[spanish]{babel}
\usepackage{tikz}
\usepackage{graphicx,enumitem}
\usepackage{multicol}
\setlength{\marginparsep}{12pt} \setlength{\marginparwidth}{0pt} \setlength{\headsep}{.8cm} \setlength{\headheight}{15pt} \setlength{\labelwidth}{0mm} \setlength{\parindent}{0mm} \renewcommand{\baselinestretch}{1.15} \setlength{\fboxsep}{5pt} \setlength{\parskip}{3mm} \setlength{\arraycolsep}{2pt}
\renewcommand{\sin}{\operatorname{sen}}
\newcommand{\N}{\ensuremath{\mathbb{N}}}
\newcommand{\Z}{\ensuremath{\mathbb{Z}}}
\newcommand{\Q}{\ensuremath{\mathbb{Q}}}
\newcommand{\R}{\ensuremath{\mathbb{R}}}
\newcommand{\C}{\ensuremath{\mathbb{C}}}
\newcommand{\I}{\ensuremath{\mathbb{I}}}
\graphicspath{{../imagenes/}{imagenes/}{..}}
\allowdisplaybreaks{}
\raggedbottom{}
\setlength{\topskip}{0pt plus 2pt}
\newcommand{\profesor}{Edward Y. Villegas}
\newcommand{\asignatura}{M\'ETODOS NUM\'ERICOS}
\DeclareMathOperator{\sen}{sen}
\renewcommand{\sin}{\sen}
\begin{document}
  \pagestyle{empty}
  \begin{minipage}{\linewidth}
    \centering
    \begin{tikzpicture}[very thick,font=\small]
      \node at (2,6) {\includegraphics[width=3.5cm]{logoudem}};
      \node at (9.5,6) {\includegraphics[width=9cm]{cbudem}};
      \node[fill=white,draw=white,inner sep=1mm] at (9.5,5.05) {\bf Permanencia con calidad, Acompa\~nar para exigir};
      \node[fill=white,draw=white,inner sep=1mm] at (7.5,4.2) {\Large\bf DEPARTAMENTO DE CIENCIAS B\'ASICAS};
      \draw (0,0) rectangle (18,3.5);
      \draw (0,2.5)--(18,2.5) (0,1.5)--(18,1.5) (15,4.2)--(18,4.2) node[below,pos=.5] {CALIFICACI\'ON} (15,2.5)--(15,7)--(18,7)--(18,3.5) (8.4,0)--(8.4,1.5) (15,0)--(15,1.5) (10,1.5)--(10,2.5);
      \node[right] at (0,3.2) {\bf Alumno:}; \node[right] at (15,3.2) {\bf Carn\'e:};
      \node[right] at (0,2.2) {\bf Asignatura:};
      \node at (6,1.95) {\asignatura};
      \node[right] at (10,2.2) {\bf Profesor:};
      \node at (15,1.95) {\profesor};
      \node[right] at (0,1.2) {\bf Examen:};
      \draw (3.8,.9) rectangle (4.4,1.3); \node[left] at (3.8,1.1) {Parcial:};
      \draw (3.8,.2) rectangle (4.4,.6); \node[left] at (3.8,.4) {Previa:};
      \draw (7.4,.9) rectangle (8,1.3); \node[left] at (7.4,1.1) {Final:};
      \draw (7.4,.2) rectangle (8,.6); \node[left] at (7.4,.4) {Habilitaci\'on:};
      \node at (4, .4) {X}; % Quiz
      \node[right] at (10,.5) {7}; % Número de grupo
      \node[right] at (10,1.) {14 de mayo de 2015}; % Fecha de presentación
      \node[right] at (8.4,1.05) {\bf Fecha:}; \node[right] at (8.4,.45) {\bf Grupo:};
      \node[align=center,text width=3cm,font=\footnotesize] at (16.5,.75) {\centering\bf Use solo tinta\\y escriba claro};
    \end{tikzpicture}
  \end{minipage}
  \begin{enumerate}[leftmargin=*,widest=9]
    \item \textbf{Punto fijo} Dado el polinomio $P(x) = x^3 - 2x + 1$ en el dominio $[0, 2]$.
    \begin{itemize}
    \item[$0.5$] ¿Es posible encontrar la integral exacta de $P(x)$ con una forma simple de la regla de Simpson? Indique la forma simple de Simpson que generaría el resultado exacto o más aproximado (en caso de no ser posible la exacta).
    \vspace{1cm}
    \item[$1.0$] Dado el tamaño de paso $h= 0.5$, determine el valor de la integral de $P(x)$ con la forma de Simpson adecuada en el intervalo indicado.
    \vspace{4cm}
    \end{itemize}
 \item Dada la ecuación diferencial $$ \frac{dy}{dt} = f(t, y), \quad y(0)= 2,$$ en el dominio $t \in \left[0, \frac{\pi}{2} \right]$ y $y \in [0, \infty]$ con $f(t, y) = \sin(t) \exp(-y)$.
    \begin{itemize}
    \item[$1.0$] Determine la constante de Lipschitz para $f(t, y)$ en el dominio mencionado.
    \vspace{6cm}
    \item[$0.5$] ¿El problema de valor inicial ilustrado posee solución única?
    \vspace{.5cm}
    \end{itemize}
    \item Dado el problema de valor inicial $$ \frac{dy}{dt} = t\exp(-t),\quad y(0)=1. $$
    \begin{itemize}
    \item[$1.0$] Determine la forma iterativa del método de Taylor de orden 3 para la solución del problema de valor inicial (sustituya las respectivas derivadas de $f(t, y)$) para $y_{i+1}$.
    \vspace{6cm}
    \item[$1.0$] Usando $h=0.2$, encuentre una estimación al valor de $y(2h)$ usando el método de Taylor orden 3.
    \end{itemize}
  \end{enumerate}
\end{document}
